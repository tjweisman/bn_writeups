\documentclass{article}
\usepackage{amsmath}
\usepackage{amsfonts}
\usepackage{amsthm}
\usepackage{enumitem}
\usepackage{dsfont}
\usepackage{amssymb}
\usepackage{pifont}
\usepackage{mathtools}

\newtheorem{thm}{Theorem}[section]
\newtheorem{lem}[thm]{Lemma}
\newtheorem{prop}[thm]{Proposition}
\newtheorem{conj}[thm]{Conjecture}
\newtheorem{cor}[thm]{Corollary}

\newtheorem*{prop*}{Proposition}

\theoremstyle{definition}
\newtheorem{claim}[thm]{Claim}
\newtheorem{defn}[thm]{Definition}
\newtheorem{quest}[thm]{Question}
\newtheorem{remark}[thm]{Remark}
\newtheorem{fact}[thm]{Fact}
\newtheorem{note}[thm]{Note}

\newtheorem*{claim*}{Claim}
\newtheorem*{quest*}{Question}
\newtheorem*{remark*}{Remark}
\newtheorem*{fact*}{Fact}

\newcommand{\N}{\ensuremath{\mathbb{N}}}
\newcommand{\Z}{\ensuremath{\mathbb{Z}}}
\newcommand{\Q}{\ensuremath{\mathbb{Q}}}
\newcommand{\R}{\ensuremath{\mathbb{R}}}
\newcommand{\C}{\ensuremath{\mathbb{C}}}
\newcommand{\F}{\ensuremath{\mathbb{F}}}
\newcommand{\AP}{\ensuremath{\mathcal{A}_{2^{n}}}}
\newcommand{\BP}{\ensuremath{\mathcal{B}_{2^{n}}}}
\newcommand{\CP}{\ensuremath{\mathcal{C}_{2^{n}}}}
\newcommand{\DP}{\ensuremath{\mathcal{D}_{2^{n}}}}
\newcommand{\EP}{\ensuremath{\mathcal{E}_{2^{n}}}}
\newcommand{\FP}{\ensuremath{\mathcal{F}_{2^{n}}}}

\newcommand{\E}{\ensuremath{\mathbb{E}}}
\newcommand{\1}{\ensuremath{\mathds{1}}}

\newcommand{\pair}[2]{\ensuremath{\langle #1, #2 \rangle}}

\DeclareMathOperator{\Gal}{Gal}
\DeclareMathOperator{\Jac}{Jac}
\DeclareMathOperator{\Var}{Var}
\DeclareMathOperator{\Cov}{Cov}
\DeclareMathOperator{\Div}{Div}
\DeclareMathOperator{\Prin}{Prin}        
\DeclareMathOperator{\im}{im}
\DeclareMathOperator{\val}{val}

\begin{document}

\begin{defn}
  Let $C = \{c_1, \ldots, c_k\}$ be a multiset of $k$ positive
  integers. Define the quantities $S(C)$ and $P(C)$ by:

  \begin{align}
    S(C) &= \sum_{i=1}^kc_1 \ldots \hat{c_i} \ldots c_k\\
    P(C) &= \prod_{i=1}^kc_i
  \end{align}

\end{defn}

\begin{lem}
  \label{decompose_q_non}
  Let $q$ be a prime with $q \equiv 3 \pmod 4$. Then for any $k$
  with $\left( \frac{k}{q} \right) = -1$, there exists $0 < m < q$
  such that $m(q-m) \equiv k \pmod q$. 
\end{lem}
\begin{proof}
  Let the set of nonresidues modulo $q$ be $R_q$. Consider the map
  $\phi:\F_q \to \F_q$ given by $\phi(x) = -x^2$. The image of
  $\phi$ is a subset of $R_q$, since (by quadratic reciprocity)
  $-1$ is a nonresidue modulo $q$. Furthermore, since the polynomial
  $-x^2 - a$ has at most two roots in $\F_q$ (for any $a$), and
  $|R_q| = (q - 1)/2$, $\phi_k$ is onto $R_q$. 

  Since $k \in R_q$, choose $m$ such that $\phi(m) = k$, and note
  that $k \equiv -m^2 \equiv m(q-m) \pmod q$, as required.
\end{proof}
\begin{lem}
  \label{decompose_q_res}
  Let $q$ be a prime with $q \equiv 1 \pmod 4$. Then for any $k$
  with $\left( \frac{k}{q} \right) = 1$, there exists $0 < m < q$
  such that $m(q-m) \equiv k \pmod q$.
\end{lem}
\begin{proof}
  Define $\phi$ as in Lemma \ref{decompose_q_non}, and note that in
  this case quadratic reciprocity implies that $\phi$ is onto the set
  of residues modulo $q$ (since $-1$ is a residue mod $q$). $k$ is
  again in the image of $\phi$, so its decomposition can be
  constructed as before.
\end{proof}
\begin{prop}
  \label{exist_q}
  Let $p$ be a prime with $p \equiv 1 \pmod 4$, and let $r$ be an
  integer with $r > 1$. Then there exists a prime $q$, with $\left(
    \frac{q}{p^r} \right) = -1$, and an integer $0 < m < q$ such that
  the quantity
  \[\frac{p^r - m(q-m)}{q}\] 
  is a positive integer.
\end{prop}
\begin{proof}
  First suppose $r$ is odd. We may assume that there exists a prime
  $q \equiv 3 \pmod 4$ such that $\left( \frac{q}{p} \right) = -1$
  and $q < p$ (citation needed). 

  Since $r$ is odd, and $p \equiv 1 \pmod 4$, $p^r$ is a nonresidue
  modulo $q$. Apply Lemma \ref{decompose_q_non} to find $m$ such
  that $p^r \equiv m(q-m) \pmod q$. Since $q < p$ and $r > 1$,
  $m(q-m) < p^r$. Therefore $p^r - m(q-m)$ is positive and divisible
  by $q$, as required.

  If $r$ is even, we may suppose there exists a prime $q$ with $q
  \equiv 1 \pmod 4$ such that $\left( \frac{q}{p} \right) = -1$ and $q
  < p$ (citation needed). Since $r$ is even, and $p \equiv 1
  \pmod 4$, $p^r$ is a residue modulo $q$. Apply Lemma
  \ref{decompose_q_res} to find an $m < q$ such that $p^r - m(q-m)$ is
  divisible by $q$, and as in the odd case, we must have $p^r - m(q-m)
  > 0$.
\end{proof}
\begin{thm}
  \label{exist_decomp}
  Let $p$ be an odd prime, and let $r > 1$. Then there exists a
  multiset of positive integers $C = \{c_1, \ldots c_k\}$ such that
  $S(C) = p^r$ and $P(C)$ is a nonresidue modulo $p^r$.
\end{thm}
\begin{proof}
  First consider the case that $p \equiv 3 \pmod 4$. Choose $k = 2$
  and $C = \{1, p^r-1\}$. Clearly $S(C) = p^r$, and by quadratic
  reciprocity, $P(C) \equiv -1 \pmod p^r$ is a nonresidue modulo
  $p^r$.

  In the case that $p \equiv 1 \pmod 4$, choose $q,m$ as in
  Proposition \ref{exist_q}. Let $k=3$, and choose 

  \[c_1 = m, \quad c_2 =  q-m, \quad c_3 = \frac{p^r - m(q-m)}{q}\]

  It is easy to check that $S(C) = p^r$.

  The quantity $P(C) = c_1c_2c_3$ is a nonresidue mod $p^r$ iff
  $\dfrac{(-1)(m(q-m))^2}{q}$ is a nonresidue mod $p$. Since $p \equiv
  1 \pmod 4$, the numerator of this expression is a residue and hence
  $\left( \frac{P(C)}{p^r} \right) = \left( \frac{q}{p^r} \right) =
  -1$, as required.
\end{proof}

\begin{remark} For $r=1$, the proof of Proposition \ref{exist_q} (and
  hence for Theorem \ref{exist_decomp}) only fails when we require the
  existence of some prime $q \equiv 3 \pmod 4$ such that $\left(
    \frac{q}{p} \right) = -1$, and $m(q-m) < p^r$ for all $m$.

  To prove Theorem \ref{exist_decomp} for the case $r=1$, it would
  thus be sufficient to show that for any prime $p \equiv 1 \pmod 4$,
  the smallest nonresidue modulo $p$ that is equivalent to $3 \pmod 4$
  is less than $2\sqrt{p}$. Computer search has verified that such a
  bound holds for all primes $p < 10^9$. The best known unconditional
  bound on such a prime $q$ is $O(p^{1/2 + \epsilon})$ for any
  $\epsilon$, but bounds that are sufficient to prove the theorem in
  this case hold under the assumption of the Generalized Riemann
  Hypothesis\footnote{citation needed}.
\end{remark}

\begin{prop}
\label{wedge_product}
Let $G_1$, $G_2$ be graphs, and let $G_{12}$ be the graph made by
adjoining $G_1$ and $G_2$ together at a single vertex ($G_1 \wedge G_2
= G_{12}$). Then

  $$\Jac(G_{12}) \simeq \Jac(G_1) \oplus \Jac(G_2)$$

  where $\Jac(G)$ here denotes the Jacobian of the graph $G$ with the
  associated monodromy pairing, and $\oplus$ denotes the orthogonal sum
  on groups with pairing.
\end{prop}

\begin{proof} We should probably actually type up the proof of this
  fact at some point. It sure would be a nice thing to have.
\end{proof}

\begin{thm}[Cite wherever this came from, and find a nicer way to say
  it]
  Let $\Gamma$ be a finite abelian group with pairing. Then
  \begin{equation}
    \Gamma \simeq \bigoplus_{i=1}^k \Z/p_i^{r_i}\Z
  \end{equation}

  for (not necessarily distinct) primes $p_i$ and positive integers
  $r_i$. 
\end{thm}

Classi

\begin{thm}[Cite this one too.]
  Let $b:\Gamma \times \Gamma \to \Q/\Z$ be a pairing on $\Gamma
  \simeq \Z/p^r\Z$ for some prime $p$ and positive integer $r$. 

  Then $b$ is isomorphic to one of the following two pairings:
  
  \begin{itemize}
    \item $b(x,y) = \frac{xy}{p^r}$
    \item $b(x,y) = \frac{axy}{p^r}$, where $a$ is any nonresidue
      modulo $p$.
  \end{itemize}
\end{thm}

\begin{thm}
  Let $(\Gamma, b)$ be a finite abelian group with pairing with group
  $\Gamma$ and pairing $b$. Suppose $|\Gamma|$ is odd. Then there
  exists a graph $G$ such that $(\Jac(G), d) \simeq (\Gamma, b)$,
  where $d$ is the monodromy pairing on $\Jac(G)$. 
\end{thm}


\end{document}