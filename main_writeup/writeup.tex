\documentclass{article}

\usepackage{amsmath}
\usepackage{amsfonts}
\usepackage{amsthm}
\usepackage{enumitem}
\usepackage{dsfont}
\usepackage[margin=1in]{geometry}
\usepackage{amssymb}
\usepackage{pifont}
\usepackage{mathtools}

\newtheorem{thm}{Theorem}
\newtheorem{lem}{Lemma}
\newtheorem{prop}{Proposition}
\newtheorem{conj}{Conjecture}
\newtheorem{cor}{Corollary}

\theoremstyle{definition}
\newtheorem{claim}{Claim}
\newtheorem{fact}{Fact}
\newtheorem{defn}{Definition}
\newtheorem*{remark}{Remark}

\newcommand{\N}{\ensuremath{\mathbb{N}}}
\newcommand{\Z}{\ensuremath{\mathbb{Z}}}
\newcommand{\Q}{\ensuremath{\mathbb{Q}}}
\newcommand{\R}{\ensuremath{\mathbb{R}}}
\newcommand{\C}{\ensuremath{\mathbb{C}}}
\newcommand{\F}{\ensuremath{\mathbb{F}}}
\newcommand{\AP}{\ensuremath{\mathcal{A}_{2^{n}}}}
\newcommand{\BP}{\ensuremath{\mathcal{B}_{2^{n}}}}
\newcommand{\CP}{\ensuremath{\mathcal{C}_{2^{n}}}}
\newcommand{\DP}{\ensuremath{\mathcal{D}_{2^{n}}}}
\newcommand{\EP}{\ensuremath{\mathcal{E}_{2^{n}}}}
\newcommand{\FP}{\ensuremath{\mathcal{F}_{2^{n}}}}

\newcommand{\E}{\ensuremath{\mathbb{E}}}
\newcommand{\1}{\ensuremath{\mathds{1}}}

\DeclareMathOperator{\Gal}{Gal}
\DeclareMathOperator{\Jac}{Jac}
\DeclareMathOperator{\Var}{Var}
\DeclareMathOperator{\Cov}{Cov}
\DeclareMathOperator{\Div}{Div}

\begin{document}

Unless stated otherwise, we will take a ``graph'' to mean a finite
connected multigraph with no loops.

It is natural to ask which finite abelian groups appear as Jacobians
of graphs. The problem can be considerably reduced by applying the
following lemma:

\begin{lem}
\label{lem:wedge_product}
  Let $G_1$ and $G_2$ be graphs. Then $\Jac(G_1 \vee G_2) = \Jac(G_1)
  \times \Jac(G_2)$. 
\end{lem}
\begin{proof}
  ...Proof goes here...
\end{proof}

Lemma \ref{lem:wedge_product}, together with the classification
theorem for finite abelian groups, tells us that if, for all $n$,
there exists a graph $G$ such that $\Jac(G)$ is cyclic of order $n$,
then \emph{all} finite abelian groups are the Jacobian of some graph.

For given $n$, we can give two possible constructions of $G$ with
$\Jac(G) \simeq \Z/n\Z$.

\begin{defn}
  $B_n$, the \emph{banana graph on $n$ edges}, is the graph with
  $V(B_n) = \{v_1, v_2\}$ and edge set consisting of $n$ copies of
  $\{v_1, v_2\}$.
\end{defn}

\begin{prop}
  \label{prop:banana_cyclic}
  Let $B_n$ be the banana graph on $n$ edges. Then $\Jac(B_n) \simeq \Z/n\Z$.
\end{prop}

\begin{proof}
  A spanning tree on $B_n$ consists of a single edge between its
  vertices $v_1$ and $v_2$, so $B_n$ has $n$ spanning trees and
  $|\Jac(B_n)| = n$. To show that $\Jac(B_n) \simeq \Z/n\Z$, it
  suffices to find a single element of order $n$.

  Let $D \in \Div^0(B_n)$ be the divisor $(v_2) - (v_1)$, and consider
  $\overline{D} = [D] \in \Jac(B_n)$. For any $1 \le m < n$, the
  divisor $mD$ is $v_1$-reduced---since $v_2$ has degree $n$, it is
  impossible to fire it without sending it into debt. If $m=n$, then
  firing $v_2$ yields the zero divisor. $m\overline{D}$ is therefore
  trivial iff $m$ is a multiple of $n$, and hence $\overline{D}$ has
  order $n$ as required.\\
\end{proof}

Banana graphs are only realizable as a multigraphs. So, we might also
wish to find a construction for a simple graph $G$ with $\Jac(G)
\simeq \Z/n\Z$, for given $n$. This is (almost) always possible, by
the following proposition:

\begin{prop}
  Let $C_n$ be the cycle graph on $n$ vertices. Then $\Jac(C_n) \simeq \Z/n\Z$.
\end{prop}

\begin{proof}
  Each spanning tree on $C_n$ corresponds to a choice of $n-1$ edges,
  so $\Jac(C_n)$ has order $n$ and it again suffices to find an
  element of order $n$. Enumerate the vertices of $C_n$ as $\{v_1,
  \ldots, v_n\}$, with $v_i, v_{i+1}$ adjacent for all $1 \le i < n$,
  and consider the divisor $v_2 - v_1$.   
\end{proof}



\begin{lem}
  \label{lem:jac_generators}
  Let $G$ be a graph. $\Jac(G)$ is generated by the set \[\{[(v) -
  (w)]: \{v, w\} \in E(G)\}\]
\end{lem}

\begin{proof}
  In general, $\Div^0(G)$ is generated by divisors of the form $(v_a)
  - (v_b)$ for $v_a, v_b \in V(G)$. Since $G$ is connected, any such
  divisor may be represented as a sum of divisors of the form $(v) -
  (w)$ for adjacent $v$ and $w$ (formal proof can be given if
  necessary). Therefore the Jacobian is generated by the equivalence
  classes of these divisors.  
\end{proof}

\end{document}