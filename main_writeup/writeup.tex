\documentclass{amsart}

\usepackage{amsmath}
\usepackage{amsfonts}
\usepackage{amsthm}
\usepackage{enumitem}
\usepackage{dsfont}
\usepackage{amssymb}
\usepackage{pifont}
\usepackage{mathtools}

\newtheorem{thm}{Theorem}[section]
\newtheorem{lem}[thm]{Lemma}
\newtheorem{prop}[thm]{Proposition}
\newtheorem{conj}[thm]{Conjecture}
\newtheorem{cor}[thm]{Corollary}

\newtheorem*{prop*}{Proposition}

\theoremstyle{definition}
\newtheorem{claim}[thm]{Claim}
\newtheorem{defn}[thm]{Definition}
\newtheorem{quest}[thm]{Question}
\newtheorem{remark}[thm]{Remark}
\newtheorem{fact}[thm]{Fact}
\newtheorem{note}[thm]{Note}

\newtheorem*{claim*}{Claim}
\newtheorem*{quest*}{Question}
\newtheorem*{remark*}{Remark}
\newtheorem*{fact*}{Fact}

\newcommand{\N}{\ensuremath{\mathbb{N}}}
\newcommand{\Z}{\ensuremath{\mathbb{Z}}}
\newcommand{\Q}{\ensuremath{\mathbb{Q}}}
\newcommand{\R}{\ensuremath{\mathbb{R}}}
\newcommand{\C}{\ensuremath{\mathbb{C}}}
\newcommand{\F}{\ensuremath{\mathbb{F}}}
\newcommand{\AP}{\ensuremath{\mathcal{A}_{2^{n}}}}
\newcommand{\BP}{\ensuremath{\mathcal{B}_{2^{n}}}}
\newcommand{\CP}{\ensuremath{\mathcal{C}_{2^{n}}}}
\newcommand{\DP}{\ensuremath{\mathcal{D}_{2^{n}}}}
\newcommand{\EP}{\ensuremath{\mathcal{E}_{2^{n}}}}
\newcommand{\FP}{\ensuremath{\mathcal{F}_{2^{n}}}}

\newcommand{\E}{\ensuremath{\mathbb{E}}}
\newcommand{\1}{\ensuremath{\mathds{1}}}

\newcommand{\pair}[2]{\ensuremath{\langle #1, #2 \rangle}}

\DeclareMathOperator{\Gal}{Gal}
\DeclareMathOperator{\Jac}{Jac}
\DeclareMathOperator{\Var}{Var}
\DeclareMathOperator{\Cov}{Cov}
\DeclareMathOperator{\Div}{Div}
\DeclareMathOperator{\Prin}{Prin}        
\DeclareMathOperator{\im}{im}
\DeclareMathOperator{\val}{val}

\newcommand{\bv}[1]{\widehat{\mathbf{#1}}}

\begin{document}

Unless stated otherwise, we will take a ``graph'' to mean a finite
connected multigraph with no loops.


\section{Jacobians of Finite Graphs}
It is natural to ask which finite abelian groups appear as Jacobians
of graphs. The problem can be considerably reduced by applying the
following lemma:

\begin{lem}
\label{lem:wedge_product}
Let $G_1$ and $G_2$ be graphs. Then $\Jac(G_1 \vee G_2) \simeq
\Jac(G_1) \times \Jac(G_2)$.
\end{lem}
\begin{proof}
  ...Proof goes here...
\end{proof}

Note that in the case that $G_1$ is a tree (and therefore has just one
spanning tree), $\Jac(G_1 \vee G_2) \simeq \Jac(G_2)$. As a result, we
have the following corollary:
\begin{cor}
  \label{cor:1_valent}
  Let $G$ be a graph, and let $A = \{v \in V(G) : \val{v} > 1\}$. If
  $G'$ is the subgraph of $G$ induced by $A$, then $\Jac(G) \simeq
  \Jac(G')$.
\end{cor}

Lemma \ref{lem:wedge_product}, together with the classification
theorem for finite abelian groups, tells us that if, for all $n$,
there exists a graph $G$ such that $\Jac(G)$ is cyclic of order $n$,
then \emph{all} finite abelian groups are the Jacobian of some graph.

For given $n$, we can give two possible constructions of $G$ with
$\Jac(G) \simeq \Z/n\Z$.

\begin{defn}
  $B_n$, the \emph{banana graph on $n$ edges}, is the graph with
  $V(B_n) = \{v_1, v_2\}$ and edge set consisting of $n$ copies of
  $\{v_1, v_2\}$.
\end{defn}

\begin{prop}
  \label{prop:banana_cyclic}
  Let $B_n$ be the banana graph on $n$ edges. Then $\Jac(B_n) \simeq \Z/n\Z$.
\end{prop}

\begin{proof}
  A spanning tree on $B_n$ consists of a single edge between its
  vertices $v_1$ and $v_2$, so $B_n$ has $n$ spanning trees and
  $|\Jac(B_n)| = n$. To show that $\Jac(B_n) \simeq \Z/n\Z$, it
  suffices to find a single element of order at least $n$.

  Let $D \in \Div^0(B_n)$ be the divisor $(v_2) - (v_1)$, and consider
  $\overline{D} = [D] \in \Jac(B_n)$. For any $1 \le m < n$, the
  divisor $mD$ is $v_1$-reduced---since $v_2$ has degree $n$, it is
  impossible to fire it without sending it into debt. If $m=n$, then
  firing $v_2$ yields the zero divisor. $m\overline{D}$ is therefore
  trivial iff $m$ is a multiple of $n$, and hence $\overline{D}$ has
  order $n$ as required.
\end{proof}

This construction is sufficient to show that all finite abelian groups
appear as the Jacobian of some graph. However, since banana graphs are
only realizable as a multigraphs, we might also wish to find a
construction for a simple graph $G$ with $\Jac(G) \simeq \Z/n\Z$ for
given $n$. This is (almost) always possible, by the following
proposition:

\begin{prop}
  \label{prop:cycle_cyclic}
  Let $C_n$ be the cycle graph on $n$ vertices. Then $\Jac(C_n) \simeq
  \Z/n\Z$.
\end{prop}

\begin{proof}
  Each spanning tree on $C_n$ corresponds to a choice of $n-1$ edges
  to include in the tree, so $\Jac(C_n)$ has order $n$ and it again
  suffices to find a single element of order at least $n$. Enumerate
  the vertices of $C_n$ as $\{v_1, \ldots, v_n\}$, with $v_i, v_{i+1}$
  adjacent for all $1 \le i < n$, and consider the divisor $D = (v_2)
  - (v_1)$. We claim that for any $1 \le m < n$, $mD$ is equivalent to
  $(v_{m+1}) - (v_1)$.

  I'm going to make either a chip-firing argument or a Laplacian
  matrix argument for why this is true, but it's pretty cumbersome so
  I'll leave it for later.

  For any $m$, the divisor $(v_{m+1}) - (v_1)$ is $q$-reduced, and
  hence $m\overline{D} = m[D]$ is nontrivial for $m < n$, as required.
\end{proof}

$C_n$ is always a simple graph for $n > 2$, so we have the following:

\begin{cor}
  Let $\Gamma$ be a finite abelian group that cannot be written as
  $\Z/2\Z \times H$ for any group $H$. Then there exists a simple
  graph $G$ such that $\Gamma \simeq \Jac(G)$.
\end{cor}

The rest of this section will investigate which groups $\Gamma =
\Z/2\Z \times H$ can possibly occur as the Jacobian of some simple
graph. It is known that there is no simple graph $G$ with $\Jac(G)
\simeq \Z/2\Z$. We extend this result as follows:

\begin{thm}
  \label{thm:2group}
  For any $k \ge 1$, if $G$ is a graph such that $\Jac(G) \simeq
  (\Z/2\Z)^k$, then $G$ is not simple.
\end{thm}

\begin{thm}
  \label{thm:2group_product}
  Let $H$ be a finite abelian group. There exists an integer $k_0$
  (depending on $H$) such that for all $k > k_0$ there does not exist
  a simple graph $G$ with $\Jac(G) \simeq (\Z/2\Z)^k \times H$.
\end{thm}

To prove Theorem \ref{thm:2group_product}, we will first prove it in a
special case, where $G$ is required to be biconnected. If $G$ is a
graph that is not biconnected, then there is a vertex $v_0$ such that
the subgraph $G'$ induced by $V(G) \setminus \{v_0\}$ is not
connected. $G$ is the wedge sum of the connected components of $G'$
(together with the vertex $v_0$), and so $\Jac(G)$ breaks down as a
direct product of Jacobians of subgraphs of $G$.

\begin{defn}
  Let $G$ be a graph, and let $\Gamma = \Jac(G)$. We will write $\mu$
  for the maximum order of an element of $\Gamma$, and $\delta$ for
  the maximum valency of a vertex in $G$.
\end{defn}

\begin{lem}
  \label{lem:delta_le_mu}
  If $G$ is biconnected, then $\delta \le \mu$.
\end{lem}
\begin{proof}
  Let $v$ be a vertex in $V(G)$ with valency $\delta$, and let $w$ be
  a vertex adjacent to $w$. Consider the divisor $D = (v) -
  (w)$, and let $m < \delta$. 

  We may apply Dhar's burning algorithm to check that $mD$ is
  $w$-reduced. Since $G$ is biconnected, there is a path from $w$ to
  each of the neighbors of $v$ that does not contain $v$, and hence
  each of the neighbors of $v$ is ``burned'' by the algorithm. $v$ has
  more than $m$ distinct neighbors, so it is burned as well, and hence
  $m\overline{D}$ is nontrivial, and $\overline{D}$ has order at
  least $\delta$.
\end{proof}

\begin{proof}[Proof of Theorem \ref{thm:2group}]
  Let $G$ be a biconnected graph with $\Jac(G) = (\Z/2\Z)^k$. By
  Corollary \ref{cor:1_valent}, we may assume without loss of
  generality that each vertex of $G$ has valency at least $2$. By
  Lemma \ref{lem:delta_le_mu}, each vertex of $G$ has valency exactly
  $2$, and $G = C_n$ for some $n$. We must have $n=2$, which means $G$
  cannot be simple.

  Now let $G$ be a graph that is not biconnected, with $\Jac(G) \simeq
  (\Z/2\Z)^k$. We proceed by induction on $k$.
  
  The base case $k=1$ is known. For $k > 1$, note that $\Jac(G) \simeq
  \Jac(G_1) \times \Jac(G_2)$ for connected subgraphs $G_1, G_2$ of
  $G$. Without loss of generality, $G_2$ is not a tree, and
  hence $\Jac(G_1) \simeq (\Z/2\Z)^{k_1}$ for $k_1 < k$. By the
  induction hypothesis, $G_1$ is not simple, and so neither is $G$.
\end{proof}



\section{Relationship to Jacobians of Metric Graphs}
\begin{lem}
  \label{lem:jac_generators}
  Let $G$ be a graph. $\Jac(G)$ is generated by the set \[\{[(v) -
  (w)]: \{v, w\} \in E(G)\}\]
\end{lem}

\begin{proof}
  In general, $\Div^0(G)$ is generated by divisors of the form $(v_a)
  - (v_b)$ for $v_a, v_b \in V(G)$. Since $G$ is connected, any such
  divisor may be represented as a sum of divisors of the form $(v) -
  (w)$ for adjacent $v$ and $w$ (formal proof can be given if
  necessary). Therefore the Jacobian is generated by the equivalence
  classes of these divisors.  
\end{proof}

\end{document}