\documentclass{amsart}

\usepackage{amsmath}
\usepackage{amsfonts}
\usepackage{amsthm}
\usepackage{enumitem}
\usepackage{dsfont}
\usepackage[margin=1in]{geometry}
\usepackage{amssymb}
\usepackage{pifont}
\usepackage{mathtools}

\newtheorem{thm}{Theorem}
\newtheorem{lem}{Lemma}
\newtheorem{prop}{Proposition}
\newtheorem{conj}{Conjecture}
\newtheorem{cor}{Corollary}

\theoremstyle{definition}
\newtheorem{claim}{Claim}
\newtheorem{fact}{Fact}
\newtheorem{defn}{Definition}
\newtheorem*{remark}{Remark}

\newcommand{\N}{\ensuremath{\mathbb{N}}}
\newcommand{\Z}{\ensuremath{\mathbb{Z}}}
\newcommand{\Q}{\ensuremath{\mathbb{Q}}}
\newcommand{\R}{\ensuremath{\mathbb{R}}}
\newcommand{\C}{\ensuremath{\mathbb{C}}}
\newcommand{\F}{\ensuremath{\mathbb{F}}}
\newcommand{\AP}{\ensuremath{\mathcal{A}_{2^{n}}}}
\newcommand{\BP}{\ensuremath{\mathcal{B}_{2^{n}}}}
\newcommand{\CP}{\ensuremath{\mathcal{C}_{2^{n}}}}
\newcommand{\DP}{\ensuremath{\mathcal{D}_{2^{n}}}}
\newcommand{\EP}{\ensuremath{\mathcal{E}_{2^{n}}}}
\newcommand{\FP}{\ensuremath{\mathcal{F}_{2^{n}}}}

\newcommand{\E}{\ensuremath{\mathbb{E}}}
\newcommand{\1}{\ensuremath{\mathds{1}}}

\DeclareMathOperator{\Gal}{Gal}
\DeclareMathOperator{\Jac}{Jac}
\DeclareMathOperator{\Var}{Var}
\DeclareMathOperator{\Cov}{Cov}
\DeclareMathOperator{\Div}{Div}

\begin{document}

Our approach will rely on quadratic reciprocity, and it will be
necessary to consider the cases $p \equiv 1 \pmod 4$ and $p \equiv 3
\pmod 4$ seperately. We will need to further differentiate between the
cases $r > 1$ and $r = 1$; it is in the latter case that our results
rely on the assumption of the Generalized Riemann Hypothesis.

For the $p \equiv 1 \pmod 4$ case, we will need the following
proposition:
\begin{prop}
  \label{prop:q_bound}
  For any prime $p$ and integer $r > 1$, there exist prime quadratic
  nonresidues $q_1 \equiv 1 \pmod 4, q_2 \equiv 3 \pmod 4$, such that
  $q_1$ and $q_2$ are both less than $2\sqrt{p}$.
\end{prop}
\begin{proof}
Fix $\varepsilon>0$, and let $\chi_1$ be the nontrivial character modulo 4 and $\chi_2$ be the quadratic character modulo $p$ (i.e., the Legendre symbol modulo $p$). These characters have orders $n_1=n_2=2$. We're looking for a prime $q$ with $\chi_1(q)=\chi_2(q)=-1$, so we want $\chi_1,\chi_2$ to be second roots of unity. These characters form a basis for the group of characters $\mathbb{X}$ associated with the degree 4 extension $K=\Q(i, \sqrt{p})$ of $\Q$. The conductor of this extension is $4p$. Thus letting $d_1=d_2=2$, we apply Theorem 1.7 in [CITE POLLACK] and find that there's a prime $q$ with $\chi_1(q)=\chi_2(q)=-1$ and
\[
q < \kappa_{\varepsilon}p^{1/2 + \varepsilon},
\]
where $\kappa_\varepsilon$ is a constant depending only on $\varepsilon$. 
\end{proof}

GRH gives us an analagous statement for $r=1$:
\begin{prop}[Conditional on GRH]
  \label{prop:q_bound_grh}
  For any prime $p$, there exists a prime quadratic nonresidue $q
  \equiv 3 \pmod 4$ such that $q < 2\sqrt{p}$.
\end{prop}
\begin{proof}
  Somehow this follows from a result of Oesterle. Again, I don't
  understand it.
\end{proof}

\end{document}