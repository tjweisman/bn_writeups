\documentclass{amsart}

\usepackage{amsmath}
\usepackage{amsfonts}
\usepackage{amsthm}
\usepackage{enumitem}
\usepackage{dsfont}
\usepackage[margin=1in]{geometry}
\usepackage{amssymb}
\usepackage{pifont}
\usepackage{mathtools}

\newtheorem{thm}{Theorem}
\newtheorem{lem}{Lemma}
\newtheorem{prop}{Proposition}
\newtheorem{conj}{Conjecture}
\newtheorem{cor}{Corollary}

\theoremstyle{definition}
\newtheorem{claim}{Claim}
\newtheorem{fact}{Fact}
\newtheorem{defn}{Definition}
\newtheorem*{remark}{Remark}

\newcommand{\N}{\ensuremath{\mathbb{N}}}
\newcommand{\Z}{\ensuremath{\mathbb{Z}}}
\newcommand{\Q}{\ensuremath{\mathbb{Q}}}
\newcommand{\R}{\ensuremath{\mathbb{R}}}
\newcommand{\C}{\ensuremath{\mathbb{C}}}
\newcommand{\F}{\ensuremath{\mathbb{F}}}
\newcommand{\AP}{\ensuremath{\mathcal{A}_{2^{n}}}}
\newcommand{\BP}{\ensuremath{\mathcal{B}_{2^{n}}}}
\newcommand{\CP}{\ensuremath{\mathcal{C}_{2^{n}}}}
\newcommand{\DP}{\ensuremath{\mathcal{D}_{2^{n}}}}
\newcommand{\EP}{\ensuremath{\mathcal{E}_{2^{n}}}}
\newcommand{\FP}{\ensuremath{\mathcal{F}_{2^{n}}}}

\newcommand{\E}{\ensuremath{\mathbb{E}}}
\newcommand{\1}{\ensuremath{\mathds{1}}}

\DeclareMathOperator{\Gal}{Gal}
\DeclareMathOperator{\Jac}{Jac}
\DeclareMathOperator{\Var}{Var}
\DeclareMathOperator{\Cov}{Cov}
\DeclareMathOperator{\Div}{Div}

\renewcommand{\qedsymbol}{$\blacksquare$}

\begin{document}

 \begin{defn} Let $\{c_{1}, \dots, c_{m}\}_{b}$ denote the banana with
$m$ super-edges where the $i$-th super-edge is subdivided into $c_{i}$
edges.
 \end{defn}
 
 \begin{defn} Let $\{c_{1}, \dots, c_{m}\}_{c}$ denote the cycle with
$m$ edges ordered counterclockwise where the $i$-th edge is copied
$c_{i}$ times.
 \end{defn}

 \begin{defn} Let $\mathcal{P}_{2}$ denote be the semigroup of
isomorphism classes of non degenerate symmetric bilinear pairings on
finite abelian 2-groups, under orthogonal direct sum.
%rewrite so as not to copy from Miranda
 \end{defn}

 \begin{prop} $\mathcal{P}_{2}$ is generated by the following
pairings:
  
   \[ \AP \text{ on } \Z/2^{n}\Z, r\ge 1; \langle 1,
1\rangle=\frac{1}{2^{n}}
   \]
   \[ \BP \text{ on } \Z/2^{n}\Z, r\ge 2; \langle 1,
1\rangle=\frac{-1}{2^{n}}
   \]
   \[ \CP \text{ on } \Z/2^{n}\Z, r\ge 3; \langle 1,
1\rangle=\frac{5}{2^{n}}
   \]
   \[ \DP \text{ on } \Z/2^{n}\Z, r\ge 3; \langle 1,
1\rangle=\frac{-5}{2^{n}}
   \]
   \[ \EP \text{ on } \Z/2^{n}\Z\times\Z/2^{n}\Z, r\ge 1; \langle
e_{i}, e_{j}\rangle=\begin{cases}0&\mbox{if } i=j\\
\frac{1}{2^{n}}&\mbox{if } i\neq j\end{cases}
   \]
   \[ \FP \text{ on } \Z/2^{n}\Z\times\Z/2^{n}\Z, r\ge 2; \langle
e_{i}, e_{j}\rangle=\begin{cases}\frac{1}{2^{n-1}}&\mbox{if } i=j\\
\frac{1}{2^{n}}&\mbox{if } i\neq j\end{cases}
   \]
   
where $\{e_{1}, e_{2}\}$ generate $\Z/2^{n}\Z\times\Z/2^{n}\Z$ in the
last two cases.
 \end{prop}
 \begin{proof} See [Mir]
 \end{proof}
 
 \begin{lem} For $r=\{1, 3, 5, 7\}$ all pairings on $\Z/2^{n}\Z$ of
the form $\langle x, y\rangle=\frac{(8m+r)xy}{2^{n}}\pmod\Z$ with
$m\in \N$ belong to the same isomorphism class.
 \end{lem}
 \begin{proof} All isomorphisms on pairings must be derived from group
isomorphisms. Furthermore, all isomorphisms on cyclic groups of
order$z$ are of the form $\phi:x\to bx$ with $b\in\{0, 1, \dots,
z-1\}$. To preserve the non-degeneracy of a pairing $b$ must be odd,
otherwise $\exists$ an element $x$ s.t. $\langle y, x\rangle=0,
\forall y$. Since $b$ is odd it can be put in the form $2c+1$. To see
that $\Phi$ sends a pairing of the form $\frac{(8m+r)xy}{2^{n}}$ to
another pairing of the form $\frac{(8m+r)xy}{2^{n}}$ note that
   \[ (8m+r)((2b+1)^{2}\equiv r(2b+1)^{2}\pmod 8
   \] and that $\{1, 3, 5, 7\}$ are the four odd elements $\pmod 8$,
and their square are all congruent to $1\pmod8$ we have that
   \[ (8m+r)((2b+1)^{2}\equiv r(2b+1)^{2}\equiv r1^{2}\equiv r\pmod8
   \] Thus different $r$ cannot be in the same isomorphism class, and
from Proposition 1we know there are at most four isomorphism classes
of pairings on $\Z/2^{n}\Z$ these must be the isomorphism classes.
 \end{proof}

 \begin{thm} For each of $\AP$, $\BP$, $\CP$, $\DP$ there exists a
graph $G$ such that the monodromy pairing on $G$ is the desired
pairing.
 \end{thm}
 \begin{proof}
  
   Case 1: $\AP$. The banana graph with $2^{n}$ edges has pairing
$\AP$.
  
   Case 2: $\BP$. The cycle graph on $2^{n}$ vertices has pairing
$\BP$.
  
   Case 3a: For $n$ odd take $\left\{1, 2,
\frac{2^{n}-2}{3}\right\}_{b}$. It is easy to check that the divisor
$v-w$ where $w$ is a trivalent vertex and $v$ is the adjacent vertex
on $\frac{2^{n}-2}{3}$ "super-edge" has order $2^{n}$. Using the slope
definition of the monodromy pairing, one obtains the following linear
equations,
    \[ a+b+c=2^{n}
    \]
    \[ a=2b
    \]
    \[ a+\left(\frac{2^{n}-2}{3}-1\right)(a+b)=c
    \] where $a$ is the slope on length $1$ super-edge, $b$ is the
slope on each edge of the length $2$ super edge, and $c$ is the slope
on the edge between $v$ and $w$. This has solution $a=2$, $b=1$,
$c=2^{n}-3$. Thus the pairing is $\frac{2^{n}-3}{2^{n}}$, which is
isomorphic to $\CP$ by Lemma $1$.
    
    Case 3b: For $n$ even take $\left\{1,1,1,
\frac{2^{n}-1}{3}\right\}_{b}$. It is easy to check that the divisor
$v-w$ where $w$ is a trivalent vertex and $v$ is the adjacent vertex
on $\frac{2^{n}-1}{3}$ "super-edge" has order $2^{n}$. Using the slope
definition of the monodromy pairing, one obtains the following linear
equationsm
     \[ a+b+c+d=2^{n}
     \]
     \[ a=b=c
     \]
     \[ a+\left(\frac{2^{n}-1}{3}-1 \right)(a+b+c)=d
     \] where $a$ is the slope on one of the length $1$ super-edges,
$b$ is the slope on one of the length $1$ super-edges, $c$ is the
slope on one of the length $1$ super-edges, and $d$ is the slope on
the edge between $v$ and $w$. This has solution $a=1$, $b=1$, $c=1$,
and $d=2^{n}-3$. Thus he pairing is $\frac{2^{n}-3}{2^{n}}$, which is
isomorphic to $\CP$ by Lemma $1$.
    
    Case 4a: For odd $n$ take $\left\{1, 2,
\frac{2^{n}-2}{3}\right\}_{c}$. It is easy to check that the divisor
$v-w$ where $w$ is the vertex with degree has $\frac{2^{n}+4}{3}$ and
$v$ is the vertex with degree has $\frac{2^{n}+1}{3}$ has order
$2^{n}$. Using the slope definition of the monodromy pairing, one
obtains the following linear equations,
     \[ \frac{2^{n}-2}{3}a+2b=2^{n}
     \]
     \[ a=3b
     \] where $a$ is slope on set of $\frac{2^{n}-2}{3}$ edges, and
$b$ is slope on the doubled edges. This has solution $b=1$ and
$a=3$. Thus the pairing is $\frac{3}{2^{n}}$ which is isomorphic to
$\DP$ by Lemma 1.
    
    Case 4b: For even $n$ take $\left\{1, 1, 1,
\frac{2^{n}-1}{3}\right\}$. It is easy to check that the divisor $v-w$
where $w$ is one of the vertices with degree has $\frac{2^{n}+2}{3}$
and $v$ is the other vertex with degree has $\frac{2^{n}+2}{3}$, has
order $2^{n}$. Using the slope definition of the monodromy pairing,
one obtains the following linear equations,
     \[ b+ \frac{2^{n}-1}{3}a=2^{n}
     \]
     \[ 3b=a
     \] where $a$ is the slope on $ \frac{2^{n}-1}{3}$ edges, and $b$
is slope on the single edges. This has solution $a=3$, $b=1$, leading
to the pairing is $\frac{3}{2^{n}}$ which is isomorphic to $\DP$ by
Lemma 1.
 \end{proof}






































  
\end{document}