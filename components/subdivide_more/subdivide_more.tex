\documentclass{amsart}

\usepackage{amsmath}
\usepackage{amsfonts}
\usepackage{amsthm}
\usepackage{enumitem}
\usepackage{dsfont}
\usepackage{amssymb}
\usepackage{pifont}
\usepackage{mathtools}

\newtheorem{thm}{Theorem}[section]
\newtheorem{lem}[thm]{Lemma}
\newtheorem{prop}[thm]{Proposition}
\newtheorem{conj}[thm]{Conjecture}
\newtheorem{cor}[thm]{Corollary}

\newtheorem*{prop*}{Proposition}

\theoremstyle{definition}
\newtheorem{claim}[thm]{Claim}
\newtheorem{defn}[thm]{Definition}
\newtheorem{quest}[thm]{Question}
\newtheorem{remark}[thm]{Remark}
\newtheorem{fact}[thm]{Fact}
\newtheorem{note}[thm]{Note}

\newtheorem*{claim*}{Claim}
\newtheorem*{quest*}{Question}
\newtheorem*{remark*}{Remark}
\newtheorem*{fact*}{Fact}

\newcommand{\N}{\ensuremath{\mathbb{N}}}
\newcommand{\Z}{\ensuremath{\mathbb{Z}}}
\newcommand{\Q}{\ensuremath{\mathbb{Q}}}
\newcommand{\R}{\ensuremath{\mathbb{R}}}
\newcommand{\C}{\ensuremath{\mathbb{C}}}
\newcommand{\F}{\ensuremath{\mathbb{F}}}
\newcommand{\AP}{\ensuremath{\mathcal{A}_{2^{n}}}}
\newcommand{\BP}{\ensuremath{\mathcal{B}_{2^{n}}}}
\newcommand{\CP}{\ensuremath{\mathcal{C}_{2^{n}}}}
\newcommand{\DP}{\ensuremath{\mathcal{D}_{2^{n}}}}
\newcommand{\EP}{\ensuremath{\mathcal{E}_{2^{n}}}}
\newcommand{\FP}{\ensuremath{\mathcal{F}_{2^{n}}}}

\newcommand{\E}{\ensuremath{\mathbb{E}}}
\newcommand{\1}{\ensuremath{\mathds{1}}}

\newcommand{\pair}[2]{\ensuremath{\langle #1, #2 \rangle}}

\DeclareMathOperator{\Gal}{Gal}
\DeclareMathOperator{\Jac}{Jac}
\DeclareMathOperator{\Var}{Var}
\DeclareMathOperator{\Cov}{Cov}
\DeclareMathOperator{\Div}{Div}
\DeclareMathOperator{\Prin}{Prin}        
\DeclareMathOperator{\im}{im}
\DeclareMathOperator{\val}{val}

\newcommand{\bv}[1]{\widehat{\mathbf{#1}}}

\begin{document}
\begin{thm}
  \label{subdivide_thm}
  Let $G$ be a graph with genus $g$, and suppose
  \begin{equation}
    \Jac(G) \simeq \Z/a_1\Z \times \Z/a_2\Z \times \ldots \times \Z/a_k\Z
  \end{equation}

  such that $a_i > 1$ and $a_{i-1} | a_i$ for all $1 < i \le k$. Let
  $G_m$ be the graph obtained by subdividing each edge of $G$ into $m$
  smaller edges. Then

  \begin{equation}
    \Jac(G_m) \simeq (\Z/m\Z)^{g-k} 
    \times \Z/ma_1\Z \times \ldots \times \Z/ma_k\Z
  \end{equation}

\end{thm}

\begin{prop}
  The quantity $g - k$ is always nonnegative.
\end{prop}
\begin{proof}
  We may construct the graph $G$ by adding $g$ edges to a spanning
  tree for $G$. Each time an edge is added, the $p$-rank for any prime
  $p$ changes by at most $1$, and consequently the number of invariant
  factors of $\Jac(G)$ changes by at most $1$. So the total number of
  invariant factors must be less than or equal to $g$.
\end{proof}

It suffices to prove Theorem \ref{subdivide_thm} for $m=p$ for prime
$p$, since we may achieve a subdivision into $b_1 \cdot b_2$ edges by
successively subdividing into $b_1$ and $b_2$ edges.

We have shown previously that $\Jac(G)$ is isomorphic to a subgroup of
$\Jac(G_p)$. As a consequence, if two divisors are equivalent in
$\Div^0(G)$, they are also equivalent in $\Div^0(G_p)$.

\begin{prop}
  \label{p_surjective}
  For each $D \in \Div^0(G)$, $\exists D' \in \Div^0(G_p)$ such that $p
  \cdot D' \sim D$.
\end{prop}
\begin{proof}
  Let $A \subseteq \Div^0(G)$ be the set of divisors of the form $v -
  w$ for $\{v, w\} \in E(G)$. It is clear that the elements of $A$
  generate $\Div^0(G)$, so it suffices to prove the proposition for
  all $a \in A$.
  
  $a = v_0 - v_p$ for some pair of vertices $v_0, v_p \in V(G)$
  connected by an edge. In $G_p$, $v_0$ and $v_p$ are connected by a
  path of length $p$, where each vertex in the path has degree $2$.

  Let the vertices in this path be $\{v_0, v_1, \ldots,
  v_p\}$. Under this enumeration, the Laplacian matrix $Q$ of $G_p$
  is of the form
  $$Q = \begin{bmatrix}
    k_1 & -1 \\
    -1 &  2 & -1 \\
    & -1 &  2 & -1 \\
    &    &    &  \ddots  \\
    &    &    &  -1 & k_2 \\ 
    &    &    &     &    & \ddots
  \end{bmatrix}$$

  where $k_1, k_2$ are the degrees of $v_0$ and $v_p$.
  
  Since the equivalence relation on divisors is defined by the image
  of the Laplacian map $\Delta$, the proposition is equivalent to the
  statement that there exist $\mathbf{u},\mathbf{v} \in \Z^n$ such
  that

  \begin{equation}
    \bv{v}_0 - \bv{v}_p  = p\mathbf{u} + Q\mathbf{v}
  \end{equation}

  where $\bv{v}_i$ represents the $i$th standard basis element for
  $\Z^n$ (corresponding to the divisor with formal sum $v_i$). Choose
  $\mathbf{u} = \bv{v}_0 - \bv{v}_1$, and $\mathbf{v} =
  \sum_{i=1}^{p-1}(p - i)\bv{v}_i$. It is easy to check that the
  required equality holds.
\end{proof}

Let $D \in \Div^0(G_p)$ be written $\sum_{i=1}^na_iv_i$, for $a_i \in
\Z$. The proof of Proposition \ref{p_surjective} also demonstrates
that if $a_1 = p$ and $a_j = 0$ for all $j \ne 0, p$, then $D \sim D'$
for some $D' \in \Jac(G)$. This also holds more generally:

\begin{prop}
  \label{p_to_jac}
  Let $D \in \Div^0(G_p)$ be written $\sum_{i=1}^na_iv_i$, and
  suppose that for each $v_i \in V(G_p) \setminus V(G)$, $p |
  a_i$. Then $D \sim D'$ for some $D' \in \Div^0(G)$. 
\end{prop}
\begin{proof}
  As in Proposition \ref{p_surjective}, enumerate the first $p+1$
  vertices on $G$ by letting $v_0, v_p$ be adjacent vertices in $G$,
  and $v_1$ to $v_{p-1}$ be the vertices on the path between them in
  $G_p$. It suffices to prove the proposition for $a_i = p$ for $1 \le
  i < p$ and $a_j = 0$ for all $v_j \in V(G_p) \setminus V(G), j \ne
  i$.

  We proceed by induction on $i$. The case $i=1$ follows directly from
  the proof of Proposition \ref{p_surjective} (the divisor $p \cdot
  v_0 - p \cdot v_1$ is equivalent to $v_0 - v_p \in \Div^0(G)$).

  For $i > 1$, it is clear from the form of the matrix $Q$ that the
  divisor $p \cdot v_{i-2} - 2p \cdot v_{i-1} + p \cdot v_i$ is in
  the image of the Laplacian map $\Delta$. By the induction
  hypothesis, there are divisors $D_1, D_2 \in \Div(G)$ such that
  $D_1 + p \cdot v_{i-2} \sim D_1'$ and $D_2 + p \cdot v_{i-1} \sim
  D_2'$ for $D_1', D_2' \in \Div^0(G)$. 

  We then have $-D_1' + 2D_2' \sim -D_1 + 2D_2 + p \cdot v_i$, so
  $D + p \cdot v_i \sim D'$ for $D \in \Div(G)$, $D' \in \Div^0(G)$,
  as required.
\end{proof}

\begin{cor}
  \label{quotient_cor}
  The quotient $\Jac(G_p) / \Jac(G)$ is isomorphic to $(\Z/p\Z)^g$,
  where $g$ is the genus of $G$.
\end{cor}
\begin{proof}
  We may first compute $|\Jac(G_p)|$ by noting that every possible way
  of choosing $g$ edges to remove from $G$ to form a spanning tree on
  $G$ corresponds to $p^g$ ways to remove $g$ edges from $G_p$ to form
  a spanning tree on $G_p$---for each edge we remove from $G$, we have
  a choice of $p$ edges to remove from $G_p$. Therefore $|\Jac(G_p)| =
  p^g \cdot |\Jac(G)|$, and the order of the quotient is $p^g$.

  Propositions \ref{p_surjective} and \ref{p_to_jac} together imply
  that the map $\phi_p : \Jac(G_p) \to \Jac(G_p)$ defined by
  $\phi_p(d) = p \cdot d$ has precisely $\Jac(G)$ as its image. So,
  for any $d \in \Jac(G_p) / \Jac(G)$, we must have that $|d|$ divides
  $p$, i.e. $|d| = 1$ or $|d| = p$. So the quotient is isomorphic to
  $(\Z/p\Z)^b$ for some $b$, and by the above it must be
  $(\Z/p\Z)^g$. 
\end{proof}

It follows from Proposition \ref{p_to_jac} that there is a set $D
\subset \Jac(G_p)$, $D = \{d_1, \ldots, d_k\}$ such that $p \cdot d_i$
is a generator for the invariant factor $\Z/a_i\Z$ in $\Jac(G)$.

For each $d_i$, we must have either $|d_i| = pa_i$ or $|d_i| =
a_i$. If $|d_i| = a_i$, then $\langle d_i \rangle$ is the same
invariant factor $\Z/a_i\Z$ of $\Jac(G)$ generated by $p \cdot d_i$,
and therefore $\gcd(p, a_i) = 1$. If $|d_i| = pa_i$, then $\langle d_i
\rangle \simeq \Z/pa_i\Z$.

So, if $j$ is the smallest integer such that $\gcd(p, a_i) = 1$ for
all $i \le j$, then the set $D$ generates the group

$$H \simeq \Z/a_1\Z \times \ldots \times \Z/a_j\Z \times \Z/pa_{j+1}\Z \times
\ldots \times \Z/pa_k\Z$$

Since $\Jac(G)$ is a subgroup of $H$, if $g \in \Jac(G_p)$ is not
generated by $D$, Corollary \ref{quotient_cor} implies that $g$ has
order $p$. So we have $\Jac(G_p) \simeq H \times (\Z/p\Z)^{g-k+j}$,
since there are $p^{g-k+j}$ elements in $\Jac(G_p)$ not in $H$.

However, since $a_1, \ldots, a_j$ are each coprime to $p$, we can
simplify the product of cyclic groups and write

\begin{equation}
\Jac(G_p) \simeq \Z/pa_1\Z \times \ldots \times \Z/pa_k\Z \times (\Z/p\Z)^{g-k}
\end{equation}

as required.
\end{document}