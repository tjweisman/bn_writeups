\documentclass[landscape,final,columns=3]{baposter}
\usepackage{geometry}
\geometry{left=1in,right=1in,top=1in,bottom=1in}
\usepackage{times}
\usepackage{calc}
\usepackage{graphicx}
\usepackage{amsmath}
\usepackage{amssymb}
\usepackage{relsize}
\usepackage{multirow}
\usepackage{bm}
\usepackage{color}

\usepackage{amsmath}
\usepackage{amsfonts}
\usepackage{amsthm}
\usepackage{enumitem}
\usepackage{dsfont}
\usepackage{amssymb}
\usepackage{pifont}
\usepackage{mathtools}

\newtheorem{thm}{Theorem}[section]
\newtheorem{lem}[thm]{Lemma}
\newtheorem{prop}[thm]{Proposition}
\newtheorem{conj}[thm]{Conjecture}
\newtheorem{cor}[thm]{Corollary}

\newtheorem*{prop*}{Proposition}

\theoremstyle{definition}
\newtheorem{claim}[thm]{Claim}
\newtheorem{defn}[thm]{Definition}
\newtheorem{quest}[thm]{Question}
\newtheorem{remark}[thm]{Remark}
\newtheorem{fact}[thm]{Fact}
\newtheorem{note}[thm]{Note}

\newtheorem*{claim*}{Claim}
\newtheorem*{quest*}{Question}
\newtheorem*{remark*}{Remark}
\newtheorem*{fact*}{Fact}

\newcommand{\N}{\ensuremath{\mathbb{N}}}
\newcommand{\Z}{\ensuremath{\mathbb{Z}}}
\newcommand{\Q}{\ensuremath{\mathbb{Q}}}
\newcommand{\R}{\ensuremath{\mathbb{R}}}
\newcommand{\C}{\ensuremath{\mathbb{C}}}
\newcommand{\F}{\ensuremath{\mathbb{F}}}
\newcommand{\AP}{\ensuremath{\mathcal{A}_{2^{n}}}}
\newcommand{\BP}{\ensuremath{\mathcal{B}_{2^{n}}}}
\newcommand{\CP}{\ensuremath{\mathcal{C}_{2^{n}}}}
\newcommand{\DP}{\ensuremath{\mathcal{D}_{2^{n}}}}
\newcommand{\EP}{\ensuremath{\mathcal{E}_{2^{n}}}}
\newcommand{\FP}{\ensuremath{\mathcal{F}_{2^{n}}}}

\newcommand{\E}{\ensuremath{\mathbb{E}}}
\newcommand{\1}{\ensuremath{\mathds{1}}}

\newcommand{\pair}[2]{\ensuremath{\langle #1, #2 \rangle}}

\DeclareMathOperator{\Gal}{Gal}
\DeclareMathOperator{\Jac}{Jac}
\DeclareMathOperator{\Var}{Var}
\DeclareMathOperator{\Cov}{Cov}
\DeclareMathOperator{\Div}{Div}
\DeclareMathOperator{\Prin}{Prin}        
\DeclareMathOperator{\im}{im}
\DeclareMathOperator{\val}{val}

\usepackage{graphicx}
\usepackage{multicol}

\usepackage{pgfbaselayers}
\pgfdeclarelayer{background}
\pgfdeclarelayer{foreground}
\pgfsetlayers{background,main,foreground}

\usepackage{helvet}
\usepackage{palatino}
\pagestyle{empty}
\newcommand{\captionfont}{\footnotesize}

\usepackage{xcolor}
\pagecolor[RGB]{255,254, 230}

\selectcolormodel{cmyk}

\graphicspath{{images/}}
\setlength{\columnsep}{0.7em}
\setlength{\columnseprule}{0mm}
\newcommand{\compresslist}{%
\setlength{\itemsep}{1pt}%
\setlength{\parskip}{0pt}%
\setlength{\parsep}{0pt}%
}

\begin{document}

\typeout{Poster Starts}
%by changing the values inside the curly brackets you can get new colors
\definecolor{mit}{cmyk}{0,.667,.667,.4}
\definecolor{wil}{cmyk}{0.5,1,0,.6}
\definecolor{silver}{cmyk}{0,0,0,0.3}
\definecolor{yellow}{cmyk}{0,0,0.9,0.0}
\definecolor{reddishyellow}{cmyk}{0,0.22,1.0,0.0}
\definecolor{black}{cmyk}{0,0,0.0,1.0}
\definecolor{darkYellow}{cmyk}{0,0,1.0,0.5}
\definecolor{darkSilver}{cmyk}{0,0,0,0.1}
\definecolor{lightyellow}{cmyk}{0,0,0.3,0.0}
\definecolor{lighteryellow}{cmyk}{0,0,0.1,0.0}
\definecolor{lightestyellow}{cmyk}{0,0,0.05,0.0}
\begin{poster}{
  grid=no,
  columns=3,
   colspacing=1em,
  bgColorOne=lighteryellow, %background inside the boxes
  bgColorTwo=lightestyellow, %background color
  borderColor=reddishyellow,
  headerColorOne=yellow,
  headerColorTwo=reddishyellow,
  headerFontColor=black,
  boxColorOne=lightyellow,
  boxColorTwo=lighteryellow,
  % Format of textbox
  textborder=roundedleft,
  % Format of text header
  eyecatcher=no,
  headerborder=open,
  headerheight=0.08\textheight,
  headershape=roundedright,
  headershade=plain,
  headerfont=\Large\textsf, %Sans Serif
  boxshade=plain,
  background=plain,
  linewidth=2pt
  }
  {
  }
{\bf{\textcolor{mit}{This Shit is Bananas}} %title
  % Authors
{\rm \\ \large  Louis Gaudet, Nicholas Wawrykow, and Theodore Weisman \ \ \
 \textbf{Advisor:} David Jensen\\
 SUMRY 2014,\ Yale University
  }}

  % University logo: change the CMU.jpeg to whatever picture you want and then uncomment this section
%  {
%   \makebox[8em][r]{
%      \begin{minipage}{16em}
%        \hfill
%        \includegraphics[height=6.5em]{mitsealY.eps}
%            \end{minipage}
%            \hspace{-1.35in}
%        \begin{minipage}{16em}
%        \hfill
%        \includegraphics[height=5.8em]{wilsealY.eps}
%            \end{minipage}
%         \hspace{-1.35in}
%        \begin{minipage}{16em}
%        \hfill
%        \includegraphics[height=5.8em]{cookie.eps}
%            \end{minipage}
%    }
%  }

  \tikzstyle{light shaded}=[top color=baposterBGtwo!30!white,bottom color=baposterBGone!30!white,shading=axis,shading angle=30]
     \newlength{\leftimgwidth}
     \setlength{\leftimgwidth}{0.78em+8.0em}

%%%%%%%%%%%%%%%%%%%%%%%%%%%%%%%%%%%%%%%%%%%%%%%%%%%%%%%%%%%%%%%%%%%%%%%%%%%%%%
%%% Now define the boxes that make up the poster
%%%---------------------------------------------------------------------------
%%% Each box has a name and can be placed absolutely or relatively.
%%% The only inconvenience is that you can only specify a relative position
%%% towards an already declared box. So if you have a box attached to the
%%% bottom, one to the top and a third one which should be in between, you
%%% have to specify the top and bottom boxes before you specify the middle
%%% box.
%%%%%%%%%%%%%%%%%%%%%%%%%%%%%%%%%%%%%%%%%%%%%%%%%%%%%%%%%%%%%%%%%%%%%%%%%%%%%%
    %
    % A colored circle useful as a bullet with an adjustably strong filling
    \newcommand{\colouredcircle}[1]{%
      \tikz{\useasboundingbox (-0.2em,-0.32em) rectangle(0.2em,0.32em); \draw[draw=black,fill=baposterBGone!80!black!#1!white,line width=0.03em] (0,0) circle(0.18em);}}

%%%%%%%%%%%%%%%%%%%%%%%%%%%%%%%%%%%%%%%%%%%%%%%%%%%%%%%%%%%%%%%%%%%%%%%%%%%%%%
  \headerbox{{{\bf{Definitions}}}}{name=defs,column=0,row=0}{
{}Fibonacci Numbers: $F_{n+1} = F_n + F_{n-1}$;\\
$F_1 = 1$, $F_2 = 2$, $F_3=3$, $F_4=5,\dots$.

\vspace{0.1in}

{\textcolor{purple}{\large{\bf{Zeckendorf's Theorem}}}}\\
Every positive integer can be written in a unique way as a sum of non-consecutive Fibonacci numbers.

\vspace{0.1in}

{\textcolor{purple}{\large{\bf{Lekkerkerker's Theorem}}}}\\
The average number of summands in the Zeckendorf decomposition for integers in $[F_n, F_{n+1})$ tends to $\frac{n}{\varphi^2+1} \approx 0.276n$, where $\varphi = \frac{1+\sqrt{5}}2$ is the golden mean.

 }


%%%%%%%%%%%%%%%%%%%%%%%%%%%%%%%%%%%%%%%%%%%%%%%%%%%%%%%%%%%%%%%%%%%%%%%%%%%%%%
 \headerbox{{\bf{Classification theorem}}}{name=class,column=0,below=defs}{
%%%%%%%%%%%%%%%%%%%%%%%%%%%%%%%%%%%%%%%%%%%%%%%%%%%%%%%%%%%%%%%%%%%%%%%%%%%%%%
{\textcolor{purple}{\large{\bf{Central Limit Type Theorem}}}}\\
 As $n\rightarrow \infty$, the distribution of the number of summands in the Zeckendorf decomposition for integers in $[F_n, F_{n+1})$ is Gaussian.

 \vspace{0.1in}

%\scalebox{.8}{\includegraphics{GaussianBehaviorNumFibands.eps}}
\centerline{Number of summands in $[F_{2010}, F_{2011})$}

\vspace{0.22in}

}

%%%%%%%%%%%%%%%%%%%%%%%%%%%%%%%%%%%%%%%%%%%%%%%%%%%%%%%%%%%%%%%%%%%%%%%%%%%%%%
 \headerbox{{\bf{Pairings on $\Z/p^r\Z$}}}{name=prpair,span=2,column=1,row=0}{
%%%%%%%%%%%%%%%%%%%%%%%%%%%%%%%%%%%%%%%%%%%%%%%%%%%%%%%%%%%%%%%%%%%%%%%%%%%%%%
%Generalizing from Fibonacci numbers to \emph{linearly recursive sequences with arbitrary nonnegative coefficients}:

%\begin{list}{\labelitemi}{\leftmargin=1em}

%\vspace{-0.1in}

%\item Recurrence relation:\\
%$H_{n+1}=c_1 H_n + c_2 H_{n-1} + \cdots + c_L H_{n-L+1}$\\
%for $n\geq L$.
%\item

%\vspace{-0.05in}
%Initial conditions:
%$H_1=1$ and\\
% $H_{n+1} =c_1 H_n + c_2 H_{n-1} + \cdots + c_n H_{1}+1$
% for $n< L$.
%\end{list}

%{\textcolor{purple}{\large{\bf{Generalized Zeckendorf's THM}}}}\\
%Every positive integer can be written as a unique sum $\sum a_iH_i$ with natural constraints on the $a_i$'s (e.g., cannot use the recurrence relation to remove any summands).\\

%{\textcolor{purple}{\large{\bf{Generalized Lekerkerker's THM}}}}\\
%The average number of summands in the generalized Zeckendorf decomposition for integers in $[H_n, H_{n+1})$ tends to $Cn+d$, where $C$ and $d$ are computable constants determined by the $c_i$'s. The value of $C$ is
%\vspace{-0.1in}
%$$
%\frac{\sum_{m=0}^{L-1}(s_m+s_{m+1}-1)(s_{m+1}-s_m)y^m(1)}{2\sum_{m=0}^{L-1}(m+1)(s_{m+1}-s_m)y^m(1)}$$

%\vspace{-0.1in}

%where $s_0=0$ and $s_m=c_1+c_2+\cdots+c_m$.\\

%{\textcolor{purple}{\large{\bf{Central Limit Type THM}}}}\\
%As $n\rightarrow \infty$, the distribution of the number of summands in the generalized Zeckendorf decomposition for integers in $[H_n, H_{n+1})$ is Gaussian.

}

%%%%%%%%%%%%%%%%%%%%%%%%%%%%%%%%%%%%%%%%%%%%%%%%%%%%%%%%%%%%%%%%%%%%%%%%%%%%%%
  \headerbox{{\bf{Acknowledgements}}}{name=ack,column=0,span=2,above=bottom}{
%%%%%%%%%%%%%%%%%%%%%%%%%%%%%%%%%%%%%%%%%%%%%%%%%%%%%%%%%%%%%%%%%%%%%%%%%%%%%%
  \smaller
  This research was conducted as part of the 2014 SUMRY program at Yale University.
}%

%%%%%%%%%%%%%%%%%%%%%%%%%%%%%%%%%%%%%%%%%%%%%%%%%%%%%%%%%%%%%%%%%%%%%%%%%%%%%%
  \headerbox{{\bf{Pairings on $\Z/2^r\Z$}}}{name=2rpair,column=1, span=2, above=ack,below=prpair}{
%%%%%%%%%%%%%%%%%%%%%%%%%%%%%%%%%%%%%%%%%%%%%%%%%%%%%%%%%%%%%%%%%%%%%%%%%%%%%%

\begin{multicols}{2}

The semigroup of isomorphism classes of non-degenerate, symmetric, bilinear pairings on finite abelian 2-groups, under orthogonal direct sum , $\mathcal{P}_{2}$ is generated by the following pairings:

  \[ \AP \text{ on } \Z/2^{n}\Z, r\ge 1; \langle 1,
1\rangle=\frac{1}{2^{n}}
   \]
   \[ \BP \text{ on } \Z/2^{n}\Z, r\ge 2; \langle 1,
1\rangle=\frac{-1}{2^{n}}
   \]
   \[ \CP \text{ on } \Z/2^{n}\Z, r\ge 3; \langle 1,
1\rangle=\frac{5}{2^{n}}
   \]
   \[ \DP \text{ on } \Z/2^{n}\Z, r\ge 3; \langle 1,
1\rangle=\frac{-5}{2^{n}}
   \]
   \[ \EP \text{ on } \Z/2^{n}\Z\times\Z/2^{n}\Z, r\ge 1; \langle
e_{i}, e_{j}\rangle=\begin{cases}0&\mbox{if } i=j\\
\frac{1}{2^{n}}&\mbox{if } i\neq j\end{cases}
   \]
   \[ \FP \text{ on } \Z/2^{n}\Z\times\Z/2^{n}\Z, r\ge 2; \langle
e_{i}, e_{j}\rangle=\begin{cases}\frac{1}{2^{n-1}}&\mbox{if } i=j\\
\frac{1}{2^{n}}&\mbox{if } i\neq j\end{cases}
   \]
   
where $\{e_{1}, e_{2}\}$ generate $\Z/2^{n}\Z\times\Z/2^{n}\Z$ in the
last two cases.

\

\ \ By recasting as a combinatorial problem and using generating functions and differentiating identities, we surmount the limitations inherent in the previous approaches.

\ \ We take the case of Fibonacci numbers as an example to show how our approach works.

\ \ Let $p(n,k)=$ \# $\{N\in [F_n, F_{n+1}):N$ has a $k$-summand
Zeckendorf decomposition$\}$ and $K$ be the random variable associated with $k$ with probability density $p(n,k)$.


\begin{list}{\labelitemi}{\leftmargin=1em}

\vspace{-0.1in}

\item {{\textcolor{blue}{\bf{Recurrence relation}}}}:\\
$p(n+1,k+1)=p(n,k+1)+p(n,k)$.

\vspace{-0.02in}

\item {{\textcolor{blue}{\bf{Generating function}}}}: \\
$\sum_{n,k>0}p(n,k)x^ky^n =\frac{y}{1-y-xy^2}$.

\vspace{-0.02in}

\item {{\textcolor{blue}{\bf{Partial fraction expansion}}}}: \\
$\frac{y}{1-y-xy^2} =\frac{y}{y_2(x)-y_1(x)}\left[\frac{1}{y-y_1(x)}-\frac{1}{y-y_2(x)}\right]$

where $y_1(x)$ and $y_2(x)$ are the roots of\\
$1-y-xy^2=0$.

{\bf{Coefficient of $y^n$}}: \\
$g(x)=\sum_{n,k>0}p(n,k)x^k$.

\vspace{-0.02in}

\item {{\textcolor{blue}{\bf{Differentiating identities}}}}: \\
$g(1)=F_{n+1}-F_n$,  $g'(1)=g(1)E[K]$,

$\left(xg'(x)\right)'|_{x=1}=g(1)E[K^2]$,

$\left(x\left(xg'(x)\right)'\right)'|_{x=1}=g(1)E[K^3]$, ...

Similar results hold for the random variable $K-E[K]$, namely the centralized $K$.

\vspace{-0.02in}

\item {{\textcolor{blue}{\bf{Method of moments}}}}: \\
$E[(K')^{2m}]/E[K'^2]\rightarrow (2m-1)!!$,

$E[(K')^{2m-1}]/E[K'^2]\rightarrow 0$.
\end{list}

\vspace{0.1em}

\end{multicols}

}

\end{poster}%
%
\end{document}
