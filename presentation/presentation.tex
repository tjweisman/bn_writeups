\documentclass[mathserif, serif, xcolor=dvipsnames]{beamer}

\usepackage{tikz}

\newtheorem{prop}{Proposition}

\newcommand{\N}{\ensuremath{\mathbb{N}}}
\newcommand{\Z}{\ensuremath{\mathbb{Z}}}
\newcommand{\Q}{\ensuremath{\mathbb{Q}}}
\newcommand{\R}{\ensuremath{\mathbb{R}}}

\DeclareMathOperator{\Jac}{Jac}
\DeclareMathOperator{\Div}{Div}
\DeclareMathOperator{\Prin}{Prin}        
\DeclareMathOperator{\im}{im}
\DeclareMathOperator{\val}{val}

\title{Jacobians of finite graphs}
\author{Louis Gaudet, Nicholas Wawrykow, Theodore Weisman\\
  Mentors: Daniel Corey, David Jensen, Dhruv Ranganathan}

\institute{Yale University, SUMRY 2014}

\begin{document}

\begin{frame}
  \titlepage
\end{frame}

\section{Background}
\begin{frame}
  \frametitle{Definitions}

  $G = \{V(G), E(G)\}$ is a finite, connected multigraph with no loops

  \begin{definition}
    A \emph{divisor} on $G$ is an assignment of an integer to each vertex.
  \end{definition}
  
  \pause
  Divisors form a group, under pointwise addition:

\begin{center}
\begin{tikzpicture}

%first divisor
\draw (60:1) -- (0,0) -- (0:1) -- (60:1) ;

\draw [fill] (0,0) circle [radius=0.05cm] ;
\draw [fill] (60:1) circle [radius=0.05cm] ;
\draw [fill] (0:1) circle [radius=0.05cm] ;

%first divisor labels
\node [right] at (60:1.15) {$\;0$} ;
\node [left] at (-.1,0) {$-2\;$} ;
\node [right] at (0:1.2) {$2$} ;

%second divisor
\draw [shift={(2.8cm,0cm)}] (60:1) -- (0,0) -- (0:1) -- (60:1) ;

\draw [fill,shift={(2.8cm,0cm)}] (0,0) circle [radius=0.05cm] ;
\draw [fill,shift={(2.8cm,0cm)}] (60:1) circle [radius=0.05cm] ;
\draw [fill,shift={(2.8cm,0cm)}] (0:1) circle [radius=0.05cm] ;

%second divisor labels
\node [right,shift={(2.8cm,0cm)}] at (60:1.15) {$\;5$} ;
\node [left,shift={(2.8cm,0cm)}] at (-.1,0) {$-2\;$} ;
\node [right,shift={(2.8cm,0cm)}] at (0:1.2) {$-1$} ;

%resulting divisor
\draw [shift={(5.8cm,0cm)}] (60:1) -- (0,0) -- (0:1) -- (60:1) ;

\draw [fill,shift={(5.8cm,0cm)}] (0,0) circle [radius=0.05cm] ;
\draw [fill,shift={(5.8cm,0cm)}] (60:1) circle [radius=0.05cm] ;
\draw [fill,shift={(5.8cm,0cm)}] (0:1) circle [radius=0.05cm] ;

%resulting divisor labels
\node [right,shift={(5.8cm,0cm)}] at (60:1.15) {$\;5$} ;
\node [left,shift={(5.8cm,0cm)}] at (-.1,0) {$-4\;$} ;
\node [right,shift={(5.8cm,0cm)}] at (0:1.2) {$1$} ;

%plus and equals signs
\node [shift={(1.9cm,-.5cm)}] at (0,1) {$+$} ;
\node [shift={(4.9cm,-.5cm)}] at (0,1) {$=$} ;

\end{tikzpicture}
\end{center}

\pause

Think of a divisor as a configuration of chips stacked on the vertices.

\end{frame}

\begin{frame}
  \frametitle{Chip firing} For a fixed graph $G$, we can play a
  ``chip-firing game'' with divisors on $G$:

\begin{center}
\begin{tikzpicture}

%first divisor
\draw (60:1) -- (0,0) -- (0:1) -- (60:1) ;

\draw [fill] (0,0) circle [radius=0.05cm] ;
\draw [fill] (60:1) circle [radius=0.05cm] ;
\draw [fill=red] (0:1) circle [radius=0.05cm] ;

%first divisor labels
\node [right] at (60:1.15) {$\;0$} ;
\node [left] at (-.1,0) {$-2\;$} ;
\node [right] at (0:1.2) {$2$} ;

%fire arrow
\node [shift={(2.6cm,-.5cm)}] at (0,1) {$\xrightarrow{\hspace{1.5cm}}$} ;
\node [above,shift={(2.6cm,-.5cm)}] at (0,1.15) {fire\;\;\;\;} ;
\draw [fill=red,shift={(2.9cm,1cm)}] (0,-.25) circle [radius=0.05cm] ;

%second divisor
\draw [shift={(4cm,0cm)}] (60:1) -- (0,0) -- (0:1) -- (60:1) ;

\draw [fill,shift={(4cm,0cm)}] (0,0) circle [radius=0.05cm] ;
\draw [fill,shift={(4cm,0cm)}] (60:1) circle [radius=0.05cm] ;
\draw [fill,shift={(4cm,0cm)}] (0:1) circle [radius=0.05cm] ;

%second divisor labels
\node [right,shift={(4cm,0cm)}] at (60:1.15) {$\;1$} ;
\node [left,shift={(4cm,0cm)}] at (-.1,0) {$-1\;$} ;
\node [right,shift={(4cm,0cm)}] at (0:1.2) {$0$} ;

\end{tikzpicture}
\end{center}


\begin{itemize}

\pause
\item
If you can get between two divisors $D$ and $D'$ via a sequence of
chip-firing moves, then $D,D'$ are \textbf{equivalent}, $D\sim D'$.

\pause 
\item Every divisor in an equivalence class has the same
  \emph{degree} (total number of chips on the graph).

\end{itemize}

\end{frame}

\begin{frame}
  \frametitle{The Jacobian group}
  \begin{itemize}
  \item Consider $\Div^0(G)$, the subgroup of divisors of degree $0$
    on $G$
    \pause
  \item What is the structure of non-equivalent elements of $\Div^0(G)$?
  \end{itemize}
  \pause
  \begin{definition}
    The Jacobian group of $G$ is
    \begin{equation*}
      \Jac(G) = \Div^0(G)/\sim
    \end{equation*}
    where $\sim$ is the equivalence relation given by chip-firing.
  \end{definition}
\end{frame}

\begin{frame}
  \frametitle{Example:}
  [INSERT EXAMPLE HERE]
\end{frame}

\begin{frame}
  \begin{itemize}
    \item $\Jac(G)$ is always a finite abelian group
      \pause
    \item For any graph $G$, $|\Jac(G)|$ is the number of spanning
      trees on $G$
      \pause
    \item The Jacobian of a cycle graph on $n$ vertices is $\Z/n\Z$. 
  \end{itemize}
  PICTURE HERE
\end{frame}

\begin{frame}
  \frametitle{\textbf{Our question:}} 

  Which finite abelian groups are the Jacobian of some graph?
\end{frame}

\begin{frame}
  \frametitle{The wedge sum}
  \begin{prop}
    Let $G_1$, $G_2$ be graphs. Then $\Jac(G_1 \vee G_2) \simeq
    \Jac(G_1) \times \Jac(G_2)$
  \end{prop}
  
  \pause 
    \begin{tikzpicture}

    
    \draw {(0:1) -- (120:1) -- (240:1)} -- cycle (270:1.2) node[below]
    {$G_1$};
    \foreach \theta in {0, 120, 240} {
      \fill (\theta:1) circle (2pt) ;
    } ;

    \node at (0:1.5) {\LARGE $\vee$};

    \begin{scope}[xshift=3cm]
      \draw {(0:1) -- (90:1) -- (180:1) -- (270:1)} -- cycle (270:1.2)
      node[below] {$G_2$};

      \foreach \theta in {0, 90, ...,  270} {
        \fill (\theta:1) circle (2pt) ;
      } ;
    \end{scope}

    \node at (0:4.7) {\LARGE $=$} ;

    \begin{scope}[xshift=6cm]
      \draw {(0:1) -- (120:1) -- (240:1)} -- cycle ;
      \foreach \theta in {0, 120, 240} {
        \fill (\theta:1) circle (2pt) ;
      } ;
      \begin{scope}[xshift=2cm]
        \draw {(0:1) -- (90:1) -- (180:1) -- (270:1)} -- cycle ;
        
        \foreach \theta in {0, 90, ...,  270} {
          \fill (\theta:1) circle (2pt) ;
        } ;
      \end{scope}
      \node[below] at (1,-1.2) {$G_1 \vee G_2$} ;
    \end{scope}
  \end{tikzpicture}

  $\Jac(G_1) \simeq \Z/3\Z$, $\Jac(G_2) \simeq \Z/4\Z$, $\Jac(G_1 \vee
  G_2) \simeq \Z/12\Z$.

\end{frame}

\begin{frame}
  \frametitle{\textbf{Our question:}} 

  \begin{itemize}
  \item
    Which finite abelian groups are the Jacobian of some graph?  

  \pause

  \begin{itemize}
  \item All of them!
  \end{itemize}

  \pause
\item
  Which finite abelian groups are the Jacobian of some \emph{simple}
  graph?
\end{itemize}

PICTURE OF C2
  
\end{frame}

\begin{frame}
  \begin{prop}
    There is no simple graph $G$ with $\Jac(G) \simeq \Z/2\Z$. 
  \end{prop}
  
  Reason: no simple graph with exactly two spanning trees.

  \pause 

  \vspace{2cm}

  Are there other groups we do \emph{not} get?
\end{frame}

\begin{frame}
  \frametitle{Our results}
  \begin{theorem}
  For any $k \ge 1$, there does not exist a simple graph $G$ with
  $\Jac(G) \simeq (\Z/2\Z)^k$.
  \end{theorem}
  \pause
  We can also give a complete characterization of all multigraphs with
  such Jacobian.

  \pause
    \begin{center}
    \begin{tikzpicture}

      \draw (0,0) to [out=15, in=165] (2,0) ;
      \draw (0,0) to [out=-15, in=-165] (2,0) ;

      \draw (2,0) to [out=75, in=225] (3, 1.7) ;
      \draw (2,0) to [out=45, in=255] (3, 1.7) ;

      \draw (2,0) to [out=-60, in=-240] (3, -1.7) ;

      \draw (3,-1.7) to [out=15, in=165] (5, -1.7) ;
      \draw (3,-1.7) to [out=-15, in=-165] (5, -1.7) ;

      \draw (5, -1.7) [out=60, in=210] to (6.7, 0) ;
      \draw (5, -1.7) [out=30, in=240] to (6.7, 0) ;

      \draw (6.7, 0) [out=120, in=-30] to (5, 1.7) ;
      \draw (6.7, 0) [out=150, in=-60] to (5, 1.7) ;

      \draw (6.7, 0) to (8.4, 1.7) ;

      \draw (6.7, 0) to [out=-30, in=120] (8.4, -1.7) ;
      \draw (6.7, 0) to [out=-60, in=150] (8.4, -1.7) ;

      \draw [fill] (0,0) circle [radius=.1cm] ;
      \draw [fill] (2,0) circle [radius=.1cm] ;
      \draw [fill] (3,1.7) circle [radius=.1cm] ;
      \draw [fill] (3,-1.7) circle [radius=.1cm] ;
      \draw [fill] (5,-1.7) circle [radius=.1cm] ;
      \draw [fill] (6.7, 0) circle [radius=.1cm] ;
      \draw [fill] (5,1.7) circle [radius=.1cm] ;
      \draw [fill] (8.4,1.7) circle [radius=.1cm] ;
      \draw [fill] (8.4,-1.7) circle [radius=.1cm] ;
    \end{tikzpicture}
    \end{center}

\end{frame}

\begin{frame}
  \frametitle{Our results}
  \begin{theorem}
    Let $H$ be any finite abelian group.  $(\Z/2\Z)^k \times H$ is not
    the Jacobian of any simple graph for all sufficiently large $k$.
  \end{theorem}

  \pause 

  Proof idea: 
  \begin{enumerate}
  \item bound $k$ when the graph is biconnected
    (i.e. \emph{wedge-free})
    \pause
  \item if $k$ is too large, the Jacobian factors as a product of
    Jacobians on wedge components
    \pause
  \item as $k$ continues to grow, eventually a component looks like
    $(\Z/2\Z)^\ell$ for $\ell \ge 1$
  \end{enumerate}

\end{frame}

\begin{frame}
  \frametitle{Future questions}
  \begin{itemize}
    \item Can we get a better bound on $k$?
      \pause
    \item Can we give a complete description of \emph{exactly} which
      groups are Jacobians of some simple graph?
      \pause
    \item Can we describe which groups are Jacobians of biconnected
      graphs?
  \end{itemize}
\end{frame}

\begin{frame}
  \frametitle{Acknowledgements}

  Thanks to our mentors, as well as Sam Payne and the rest of the
  SUMRY.
  
\end{frame}

\begin{frame}
  \frametitle{References}
\end{frame}

\end{document}
