\documentclass[mathserif, serif, xcolor=dvipsnames]{beamer}

\usepackage{tikz}
\usepackage{xcolor}
\usepackage{framed}

\setbeamertemplate{navigation symbols}{}

\usetheme{default}
\usecolortheme{crane}

\newtheorem{prop}{Proposition}

\newcommand{\N}{\ensuremath{\mathbb{N}}}
\newcommand{\Z}{\ensuremath{\mathbb{Z}}}
\newcommand{\Q}{\ensuremath{\mathbb{Q}}}
\newcommand{\R}{\ensuremath{\mathbb{R}}}

\DeclareMathOperator{\Jac}{Jac}
\DeclareMathOperator{\Div}{Div}
\DeclareMathOperator{\Prin}{Prin}        
\DeclareMathOperator{\im}{im}
\DeclareMathOperator{\val}{val}

\title{Jacobians of finite graphs}
\author[L. Gaudet, N. Wawrykow, T. Weisman]{Louis Gaudet, Nicholas Wawrykow, Theodore Weisman\\
  {\tiny Mentors: Daniel Corey, David Jensen, Dhruv Ranganathan}}

\institute{Yale University, SUMRY 2014}
\date{}

\begin{document}

\begin{frame}
  \titlepage
\end{frame}

%\section{Background}
\begin{frame}
  \frametitle{Definitions}

  $G = (V(G), E(G))$ is a finite, connected multigraph (no loops).
  
  \pause
  \begin{itemize}
  \item For later: a \textbf{simple} graph is one with no multi-edges.
  \end{itemize}

  \pause

  \begin{definition}
    A \textbf{divisor} on $G$ is an assignment of an integer to each vertex.
  \end{definition}
  
  \pause
  Divisors form a group, under vertex-wise addition:
  
  \pause

\begin{center}
\begin{tikzpicture}

%first divisor
\draw (60:1) -- (0,0) -- (0:1) -- (60:1) ;

\draw [fill] (0,0) circle [radius=0.06cm] ;
\draw [fill] (60:1) circle [radius=0.06cm] ;
\draw [fill] (0:1) circle [radius=0.06cm] ;

%first divisor labels
\node [right] at (60:1) {$0$} ;
\node [left] at (0,0) {$-2$} ;
\node [right] at (0:1) {$2$} ;

\pause

%second divisor
\draw [shift={(2.8cm,0cm)}] (60:1) -- (0,0) -- (0:1) -- (60:1) ;

\draw [fill,shift={(2.8cm,0cm)}] (0,0) circle [radius=0.06cm] ;
\draw [fill,shift={(2.8cm,0cm)}] (60:1) circle [radius=0.06cm] ;
\draw [fill,shift={(2.8cm,0cm)}] (0:1) circle [radius=0.06cm] ;

%second divisor labels
\node [right,shift={(2.8cm,0cm)}] at (60:1) {$5$} ;
\node [left,shift={(2.8cm,0cm)}] at (0,0) {$-2$} ;
\node [right,shift={(2.8cm,0cm)}] at (0:1) {$-1$} ;

\pause

%plus and equals signs
\node [shift={(1.9cm,-.5cm)}] at (0,1) {$+$} ;
\node [shift={(4.9cm,-.5cm)}] at (0,1) {$=$} ;

%resulting divisor
\draw [shift={(5.8cm,0cm)}] (60:1) -- (0,0) -- (0:1) -- (60:1) ;

\draw [fill,shift={(5.8cm,0cm)}] (0,0) circle [radius=0.06cm] ;
\draw [fill,shift={(5.8cm,0cm)}] (60:1) circle [radius=0.06cm] ;
\draw [fill,shift={(5.8cm,0cm)}] (0:1) circle [radius=0.06cm] ;

%resulting divisor labels
\node [right,shift={(5.8cm,0cm)}] at (60:1) {$5$} ;
\node [left,shift={(5.8cm,0cm)}] at (0,0) {$-4$} ;
\node [right,shift={(5.8cm,0cm)}] at (0:1) {$1$} ;

\end{tikzpicture}
\end{center}

{\color{white} Think of a divisor as a ``configuration of chips'' stacked on the vertices.}

\end{frame}

\begin{frame}
  \frametitle{Definitions}

  $G = (V(G), E(G))$ is a finite, connected multigraph (no loops).
 
  \begin{itemize}
  \item For later: a \textbf{simple} graph is one with no multi-edges.
  \end{itemize}

  \begin{definition}
    A \textbf{divisor} on $G$ is an assignment of an integer to each vertex.
  \end{definition}
  
  Divisors form a group, under vertex-wise addition:

\begin{center}
\begin{tikzpicture}

%first divisor
\draw (60:1) -- (0,0) -- (0:1) -- (60:1) ;

\draw [fill] (0,0) circle [radius=0.06cm] ;
\draw [fill] (60:1) circle [radius=0.06cm] ;
\draw [fill] (0:1) circle [radius=0.06cm] ;

%first divisor labels
\node [right] at (60:1) {$0$} ;
\node [left] at (0,0) {$-2$} ;
\node [right] at (0:1) {$2$} ;

%second divisor
\draw [shift={(2.8cm,0cm)}] (60:1) -- (0,0) -- (0:1) -- (60:1) ;

\draw [fill,shift={(2.8cm,0cm)}] (0,0) circle [radius=0.06cm] ;
\draw [fill,shift={(2.8cm,0cm)}] (60:1) circle [radius=0.06cm] ;
\draw [fill,shift={(2.8cm,0cm)}] (0:1) circle [radius=0.06cm] ;

%second divisor labels
\node [right,shift={(2.8cm,0cm)}] at (60:1) {$5$} ;
\node [left,shift={(2.8cm,0cm)}] at (0,0) {$-2$} ;
\node [right,shift={(2.8cm,0cm)}] at (0:1) {$-1$} ;

%plus and equals signs
\node [shift={(1.9cm,-.5cm)}] at (0,1) {$+$} ;
\node [shift={(4.9cm,-.5cm)}] at (0,1) {$=$} ;

%resulting divisor
\draw [shift={(5.8cm,0cm)}] (60:1) -- (0,0) -- (0:1) -- (60:1) ;

\draw [fill,shift={(5.8cm,0cm)}] (0,0) circle [radius=0.06cm] ;
\draw [fill,shift={(5.8cm,0cm)}] (60:1) circle [radius=0.06cm] ;
\draw [fill,shift={(5.8cm,0cm)}] (0:1) circle [radius=0.06cm] ;

%resulting divisor labels
\node [right,shift={(5.8cm,0cm)}] at (60:1) {$5$} ;
\node [left,shift={(5.8cm,0cm)}] at (0,0) {$-4$} ;
\node [right,shift={(5.8cm,0cm)}] at (0:1) {$1$} ;

\end{tikzpicture}
\end{center}

Think of a divisor as a ``configuration of chips'' stacked on the vertices.

\end{frame}

\begin{frame}
  \frametitle{Chip-firing} For a fixed graph $G$, we can play a
  \textbf{chip-firing game} with divisors on $G$:

\pause

\begin{center}
\begin{tikzpicture}

%first divisor
\draw (60:1) -- (0,0) -- (0:1) -- (60:1) ;

\draw [fill] (0,0) circle [radius=0.06cm] ;
\draw [fill] (60:1) circle [radius=0.06cm] ;
\draw [fill] (0:1) circle [radius=0.06cm] ;

%first divisor labels
\node [right] at (60:1) {$0$} ;
\node [left] at (0,0) {$-2$} ;
\node [right] at (0:1) {$2$} ;

\pause

%changes fire vertex to red
\draw [fill=red] (0:1) circle [radius=0.06cm] ;
%fire arrow
\node [shift={(2.6cm,-.5cm)}] at (0,1) {$\xrightarrow{\hspace{1.5cm}}$} ;
\node [above,shift={(2.6cm,-.5cm)}] at (0,1) {fire\;\;\;\;} ;
\draw [fill=red,shift={(2.9cm,1cm)}] (0,-.25) circle [radius=0.06cm] ;

\pause

%second divisor
\draw [shift={(4cm,0cm)}] (60:1) -- (0,0) -- (0:1) -- (60:1) ;

\draw [fill,shift={(4cm,0cm)}] (0,0) circle [radius=0.06cm] ;
\draw [fill,shift={(4cm,0cm)}] (60:1) circle [radius=0.06cm] ;
\draw [fill,shift={(4cm,0cm)}] (0:1) circle [radius=0.06cm] ;

%second divisor labels
\node [right,shift={(4cm,0cm)}] at (60:1) {$1$} ;
\node [left,shift={(4cm,0cm)}] at (0,0) {$-1$} ;
\node [right,shift={(4cm,0cm)}] at (0:1) {$0$} ;

\pause

%changes fire vertex to red
\draw [fill=red,shift={(4cm,0cm)}] (0:0) circle [radius=0.06cm] ;
%fire arrow
\node [shift={(6.6cm,-.5cm)}] at (0,1) {$\xrightarrow{\hspace{1.5cm}}$} ;
\node [above,shift={(6.6cm,-.5cm)}] at (0,1) {fire\;\;\;\;} ;
\draw [fill=red,shift={(6.9cm,1cm)}] (0,-.25) circle [radius=0.06cm] ;

\pause

%third divisor
\draw [shift={(8cm,0cm)}] (60:1) -- (0,0) -- (0:1) -- (60:1) ;

\draw [fill,shift={(8cm,0cm)}] (0,0) circle [radius=0.06cm] ;
\draw [fill,shift={(8cm,0cm)}] (60:1) circle [radius=0.06cm] ;
\draw [fill,shift={(8cm,0cm)}] (0:1) circle [radius=0.06cm] ;

%third divisor labels
\node [right,shift={(8cm,0cm)}] at (60:1) {$2$} ;
\node [left,shift={(8cm,0cm)}] at (0,0) {$-3$} ;
\node [right,shift={(8cm,0cm)}] at (0:1) {$1$} ;

\end{tikzpicture}
\end{center}


\begin{itemize}

\item[] {\color{white}
If you can get between two divisors $D$ and $D'$ via a sequence of
chip-firing moves, then $D$ and $D'$ are \textbf{equivalent}, $D\sim D'$.}

\item[] {\color{white}
Every divisor in an equivalence class has the same
  \textbf{degree} (total number of chips on the graph).}

\end{itemize}

\end{frame}

\begin{frame}
  \frametitle{Chip-firing} For a fixed graph $G$, we can play a
  \textbf{chip-firing game} with divisors on $G$:

\begin{center}
\begin{tikzpicture}

%first divisor
\draw (60:1) -- (0,0) -- (0:1) -- (60:1) ;

\draw [fill] (0,0) circle [radius=0.06cm] ;
\draw [fill] (60:1) circle [radius=0.06cm] ;
\draw [fill] (0:1) circle [radius=0.06cm] ;

%first divisor labels
\node [right] at (60:1) {$0$} ;
\node [left] at (0,0) {$-2$} ;
\node [right] at (0:1) {$2$} ;

%changes fire vertex to red
\draw [fill=red] (0:1) circle [radius=0.06cm] ;
%fire arrow
\node [shift={(2.6cm,-.5cm)}] at (0,1) {$\xrightarrow{\hspace{1.5cm}}$} ;
\node [above,shift={(2.6cm,-.5cm)}] at (0,1) {fire\;\;\;\;} ;
\draw [fill=red,shift={(2.9cm,1cm)}] (0,-.25) circle [radius=0.06cm] ;

%second divisor
\draw [shift={(4cm,0cm)}] (60:1) -- (0,0) -- (0:1) -- (60:1) ;

\draw [fill,shift={(4cm,0cm)}] (0,0) circle [radius=0.06cm] ;
\draw [fill,shift={(4cm,0cm)}] (60:1) circle [radius=0.06cm] ;
\draw [fill,shift={(4cm,0cm)}] (0:1) circle [radius=0.06cm] ;

%second divisor labels
\node [right,shift={(4cm,0cm)}] at (60:1) {$1$} ;
\node [left,shift={(4cm,0cm)}] at (0,0) {$-1$} ;
\node [right,shift={(4cm,0cm)}] at (0:1) {$0$} ;

%changes fire vertex to red
\draw [fill=red,shift={(4cm,0cm)}] (0:0) circle [radius=0.06cm] ;
%fire arrow
\node [shift={(6.6cm,-.5cm)}] at (0,1) {$\xrightarrow{\hspace{1.5cm}}$} ;
\node [above,shift={(6.6cm,-.5cm)}] at (0,1) {fire\;\;\;\;} ;
\draw [fill=red,shift={(6.9cm,1cm)}] (0,-.25) circle [radius=0.06cm] ;

%third divisor
\draw [shift={(8cm,0cm)}] (60:1) -- (0,0) -- (0:1) -- (60:1) ;

\draw [fill,shift={(8cm,0cm)}] (0,0) circle [radius=0.06cm] ;
\draw [fill,shift={(8cm,0cm)}] (60:1) circle [radius=0.06cm] ;
\draw [fill,shift={(8cm,0cm)}] (0:1) circle [radius=0.06cm] ;

%third divisor labels
\node [right,shift={(8cm,0cm)}] at (60:1) {$2$} ;
\node [left,shift={(8cm,0cm)}] at (0,0) {$-3$} ;
\node [right,shift={(8cm,0cm)}] at (0:1) {$1$} ;

\end{tikzpicture}
\end{center}

\begin{itemize}

\item
If you can get between two divisors $D$ and $D'$ via a sequence of
chip-firing moves, then $D$ and $D'$ are \textbf{equivalent}, $D\sim D'$.

\pause
\item
Every divisor in an equivalence class has the same
  \textbf{degree} (total number of chips on the graph).

\end{itemize}

\end{frame}

\begin{frame}
  \frametitle{The Jacobian group}
  \begin{itemize}
  \item $\Div^0(G)$ $=$ subgroup of divisors of degree $0$
    on $G$.
    \pause
  \item What is the structure of the equivalence classes of elements of $\Div^0(G)$?
  \end{itemize}
  \pause
  \begin{definition}
    The \textbf{Jacobian} group of $G$ is
    \begin{equation*}
      \Jac(G) = \Div^0(G)/\sim,
    \end{equation*}
    where $\sim$ is the equivalence relation given by chip-firing.
  \end{definition}
\end{frame}

\begin{frame}
  \frametitle{Example}
  
  Consider $C_3$, the cycle graph on three vertices.
  
  \pause

\begin{center}
\begin{tikzpicture}

%text
\node [left,shift={(-1cm,0cm)}] at (0,0.4) {$\Jac(C_3) =$} ;
\node [left,shift={(-.5cm,0cm)}] at (0,0.4) {$\Bigg\{$} ;
\node [right,shift={(7.5cm,0cm)}] at (0,0.4) {$\Bigg\}$} ;
\node [right,shift={(1.7cm,0cm)}] at (0,0) {\Huge ,} ;
\node [right,shift={(4.7cm,0cm)}] at (0,0) {\Huge ,} ;
\node [right,shift={(7.9cm,0.2cm)}] at (0,0) {\Huge .} ;

%first divisor
\draw (60:1) -- (0,0) -- (0:1) -- (60:1) ;

\draw [fill] (0,0) circle [radius=0.06cm] ;
\draw [fill] (60:1) circle [radius=0.06cm] ;
\draw [fill] (0:1) circle [radius=0.06cm] ;

%first divisor labels
\node [right] at (60:1) {$0$} ;
\node [left] at (0,0) {$0$} ;
\node [right] at (0:1) {$0$} ;

%second divisor
\draw [shift={(3cm,0cm)}] (60:1) -- (0,0) -- (0:1) -- (60:1) ;

\draw [fill,shift={(3cm,0cm)}] (0,0) circle [radius=0.06cm] ;
\draw [fill,shift={(3cm,0cm)}] (60:1) circle [radius=0.06cm] ;
\draw [fill,shift={(3cm,0cm)}] (0:1) circle [radius=0.06cm] ;

%second divisor labels
\node [right,shift={(3cm,0cm)}] at (60:1) {$0$} ;
\node [left,shift={(3cm,0cm)}] at (0,0) {$-1$} ;
\node [right,shift={(3cm,0cm)}] at (0:1) {$1$} ;

%third divisor
\draw [shift={(6cm,0cm)}] (60:1) -- (0,0) -- (0:1) -- (60:1) ;

\draw [fill,shift={(6cm,0cm)}] (0,0) circle [radius=0.06cm] ;
\draw [fill,shift={(6cm,0cm)}] (60:1) circle [radius=0.06cm] ;
\draw [fill,shift={(6cm,0cm)}] (0:1) circle [radius=0.06cm] ;

%third divisor labels
\node [right,shift={(6cm,0cm)}] at (60:1) {$0$} ;
\node [left,shift={(6cm,0cm)}] at (0,0) {$-2$} ;
\node [right,shift={(6cm,0cm)}] at (0:1) {$2$} ;

\end{tikzpicture}
\end{center}

\pause

\textbf{Thus:} \; $\Jac(C_3)\simeq \Z/3\Z$.

%%%%%%%%%%%%%%%%%%%%%%%
%showing that the divisor is equivalent to 0
%%%%%%%%%%%%%%%%%%%%%%%

\pause

\begin{center}
\begin{tikzpicture}

%first divisor
\draw (60:1) -- (0,0) -- (0:1) -- (60:1) ;

\draw [fill] (0,0) circle [radius=0.06cm] ;
\draw [fill] (60:1) circle [radius=0.06cm] ;
\draw [fill] (0:1) circle [radius=0.06cm] ;

%first divisor labels
\node [right] at (60:1) {$0$} ;
\node [left] at (0,0) {$-3$} ;
\node [right] at (0:1) {$3$} ;

%changes fire vertex to red
\draw [fill=red] (0:1) circle [radius=0.06cm] ;
%fire arrow
\node [shift={(2.6cm,-.5cm)}] at (0,1) {$\xrightarrow{\hspace{1.5cm}}$} ;
\node [above,shift={(2.6cm,-.5cm)}] at (0,1) {fire\quad $2\times$} ;
\draw [fill=red,shift={(2.9cm,1cm)}] (-.3,-.25) circle [radius=0.06cm] ;

%second divisor
\draw [shift={(4cm,0cm)}] (60:1) -- (0,0) -- (0:1) -- (60:1) ;

\draw [fill,shift={(4cm,0cm)}] (0,0) circle [radius=0.06cm] ;
\draw [fill,shift={(4cm,0cm)}] (60:1) circle [radius=0.06cm] ;
\draw [fill,shift={(4cm,0cm)}] (0:1) circle [radius=0.06cm] ;

%second divisor labels
\node [right,shift={(4cm,0cm)}] at (60:1) {$2$} ;
\node [left,shift={(4cm,0cm)}] at (0,0) {$-1$} ;
\node [right,shift={(4cm,0cm)}] at (0:1) {$-1$} ;

%changes fire vertex to red
\draw [fill=red,shift={(4cm,0cm)}] (60:1) circle [radius=0.06cm] ;
%fire arrow
\node [shift={(6.6cm,-.5cm)}] at (0,1) {$\xrightarrow{\hspace{1.5cm}}$} ;
\node [above,shift={(6.6cm,-.5cm)}] at (0,1) {fire\;\;\;\;} ;
\draw [fill=red,shift={(6.9cm,1cm)}] (0,-.25) circle [radius=0.06cm] ;

%third divisor
\draw [shift={(8cm,0cm)}] (60:1) -- (0,0) -- (0:1) -- (60:1) ;

\draw [fill,shift={(8cm,0cm)}] (0,0) circle [radius=0.06cm] ;
\draw [fill,shift={(8cm,0cm)}] (60:1) circle [radius=0.06cm] ;
\draw [fill,shift={(8cm,0cm)}] (0:1) circle [radius=0.06cm] ;

%third divisor labels
\node [right,shift={(8cm,0cm)}] at (60:1) {$0$} ;
\node [left,shift={(8cm,0cm)}] at (0,0) {$0$} ;
\node [right,shift={(8cm,0cm)}] at (0:1) {$0$} ;

\end{tikzpicture}
\end{center}

\end{frame}

\begin{frame}
\frametitle{Facts about $\Jac(G)$}
  \begin{itemize}
    \item $\Jac(G)$ is always a finite abelian group.
      \pause
    \item $|\Jac(G)| =$ the number of \textbf{spanning
      trees} on $G$.
      \pause
    \item For the $n$-cycle $C_n$, we have $\Jac(C_n)\simeq \Z/n\Z$. 
  \end{itemize}

\pause

\begin{center}
\begin{tikzpicture}

\draw (0,0) to [out=30,in=150] (0:2) ;
\draw (0,0) to [out=-30,in=-150] (0:2) ;
\draw [fill] (0,0) circle [radius=0.06cm] ;
\draw [fill] (0:2) circle [radius=0.06cm] ;

\begin{scope}[xshift=6cm]
      \draw {(0:1) -- (72:1) -- (144:1) -- (216:1) -- (288:1)} -- cycle (288:1.2) ;

      \foreach \theta in {0, 72, ...,  288} {
        \fill (\theta:1) circle (2pt) ;
      } ;
    \end{scope}
    
%labels
\node [below] at (1,-0.8) {$\Jac(C_2)\simeq \Z/2\Z$} ;
\node [below] at (6,-1.2) {$\Jac(C_5)\simeq \Z/5\Z$} ;

\end{tikzpicture}
\end{center}

\end{frame}

%\section{Questions and results}

\begin{frame}
  \frametitle{Our question} 

  Which finite abelian groups are the Jacobian of some graph?
\end{frame}

\begin{frame}
  \frametitle{The wedge sum}
  \begin{prop}[Cori and Rossin, 2000]
    Let $G_1$, $G_2$ be graphs. Then $\Jac(G_1 \vee G_2) \simeq
    \Jac(G_1) \times \Jac(G_2)$.
  \end{prop}
  
  \vspace{0.3cm}
  
  \pause
  \begin{center}
    \begin{tikzpicture}[scale=0.9]

    
    \draw {(0:1) -- (120:1) -- (240:1)} -- cycle (270:1.2) node[below]
    {$G_1$};
    \foreach \theta in {0, 120, 240} {
      \fill (\theta:1) circle (2pt) ;
    } ;

    \node at (0:1.5) {\LARGE $\vee$};

    \begin{scope}[xshift=3cm]
      \draw {(0:1) -- (90:1) -- (180:1) -- (270:1)} -- cycle (270:1.2)
      node[below] {$G_2$};

      \foreach \theta in {0, 90, ...,  270} {
        \fill (\theta:1) circle (2pt) ;
      } ;
    \end{scope}

    \node at (0:4.7) {\LARGE $=$} ;

    \begin{scope}[xshift=6cm]
      \draw {(0:1) -- (120:1) -- (240:1)} -- cycle ;
      \foreach \theta in {0, 120, 240} {
        \fill (\theta:1) circle (2pt) ;
      } ;
      \begin{scope}[xshift=2cm]
        \draw {(0:1) -- (90:1) -- (180:1) -- (270:1)} -- cycle ;
        
        \foreach \theta in {0, 90, ...,  270} {
          \fill (\theta:1) circle (2pt) ;
        } ;
      \end{scope}
      \node[below] at (1,-1.2) {$G_1 \vee G_2$} ;
    \end{scope}
    
    %%% wedge vertices
    \draw [fill=yellow] (0:1) circle [radius=0.08cm] ;
    \draw [fill=yellow,xshift=3cm] (180:1) circle [radius=0.08cm] ;
    \draw [fill=yellow,xshift=6cm] (0:1) circle [radius=0.08cm] ;
    
  \end{tikzpicture}
  \end{center}
 
 \pause

  $\Jac(G_1) \simeq \Z/3\Z$, \quad $\Jac(G_2) \simeq \Z/4\Z$,
  
  \pause
  \bigskip
  $\Jac(G_1 \vee G_2) \simeq \Z/12\Z$.

\end{frame}

\begin{frame}
  \frametitle{Our question} 

  \begin{itemize}
  \item
    Which finite abelian groups are the Jacobian of some graph?  

\medskip
  \pause

  \begin{itemize}
  \item All of them! (by wedging cycle graphs)
  \end{itemize}

\medskip
  \pause
\item
  Which finite abelian groups are the Jacobian of some \textbf{simple}
  graph?
  \medskip
  \pause
  \begin{itemize}
  \item We can use $C_n$ only for $n\ge3$.
   \medskip
  \item We're only interested in $\Z/2\Z\times H$ for some finite abelian group $H$. 
  \end{itemize}
  
\end{itemize}
  
\end{frame}

\begin{frame}
 \frametitle{A known result} 
  \begin{prop}
    There does not exist a simple graph $G$ with $\Jac(G) \simeq \Z/2\Z$. 
  \end{prop}
 \vspace{0.3cm}
 
 \pause
  
 \textbf{Proof.} Any simple graph with at least 2 spanning trees must have a third. $\square$

  \pause 

  \vspace{1.8cm}

  \textbf{Question:} Are there other groups we \emph{cannot} get?
\end{frame}

\begin{frame}
  \frametitle{Our results}
  \begin{theorem}[GWW]
  For any $k \ge 1$, there does not exist a simple graph $G$ with
  $\Jac(G) \simeq (\Z/2\Z)^k$.
  \end{theorem}
  
  \vspace{0.3cm}
  
  \pause
  We can also give a complete characterization of all multigraphs with
  this Jacobian.

  \pause
    \begin{center}
    \begin{tikzpicture}[scale=0.7]

      \draw (0,0) to [out=15, in=165] (2,0) ;
      \draw (0,0) to [out=-15, in=-165] (2,0) ;

      \draw (2,0) to [out=75, in=225] (3, 1.7) ;
      \draw (2,0) to [out=45, in=255] (3, 1.7) ;

      \draw (2,0) to [out=-60, in=-240] (3, -1.7) ;

      \draw (3,-1.7) to [out=15, in=165] (5, -1.7) ;
      \draw (3,-1.7) to [out=-15, in=-165] (5, -1.7) ;

      \draw (5, -1.7) [out=60, in=210] to (6.7, 0) ;
      \draw (5, -1.7) [out=30, in=240] to (6.7, 0) ;

      \draw (6.7, 0) [out=120, in=-30] to (5, 1.7) ;
      \draw (6.7, 0) [out=150, in=-60] to (5, 1.7) ;

      \draw (6.7, 0) to (8.4, 1.7) ;

      \draw (6.7, 0) to [out=-30, in=120] (8.4, -1.7) ;
      \draw (6.7, 0) to [out=-60, in=150] (8.4, -1.7) ;

      \draw [fill] (0,0) circle [radius=.1cm] ;
      \draw [fill] (2,0) circle [radius=.1cm] ;
      \draw [fill] (3,1.7) circle [radius=.1cm] ;
      \draw [fill] (3,-1.7) circle [radius=.1cm] ;
      \draw [fill] (5,-1.7) circle [radius=.1cm] ;
      \draw [fill] (6.7, 0) circle [radius=.1cm] ;
      \draw [fill] (5,1.7) circle [radius=.1cm] ;
      \draw [fill] (8.4,1.7) circle [radius=.1cm] ;
      \draw [fill] (8.4,-1.7) circle [radius=.1cm] ;
      
      \node at (4.2,-3) {$\Jac(G)\simeq (\Z/2\Z)^6$} ;
    \end{tikzpicture}
    \end{center}

\end{frame}

\begin{frame}
  \frametitle{Our results}
  \begin{theorem}[GWW]
    Let $H$ be any finite abelian group. Then there exists $k_H$ such that for all $k>k_H$, $(\Z/2\Z)^k \times H$ is not the Jacobian of any simple graph. 
  \end{theorem}
  \vspace{0.6cm}

  \pause 

  \textbf{Proof idea:}
  \begin{enumerate}
  \item Bound $k$ when the graph is \textbf{biconnected}
    (i.e., wedge-free).
    \pause
  \item If $k$ is too large, the Jacobian factors as a product of
    Jacobians on wedge components.
    \pause
  \item As $k$ continues to grow, eventually a component looks like
    $(\Z/2\Z)^\ell$ for $\ell \ge 1$.
  \end{enumerate}

\end{frame}

\begin{frame}
  \frametitle{Our results}
  \begin{theorem}[GWW]
    Let $G$ be a simple, biconnected graph with $\Jac(G)\simeq(\Z/3\Z)^k$ with $k\ge1$. Then in fact $G=C_3$ and $k=1$. 
  \end{theorem}

 \vspace{0.3cm}
  
  \pause
  We can also give a complete characterization of all simple graphs with
  this Jacobian.

  \pause
    \begin{center}
    \begin{tikzpicture}[scale=1.1]

	%triangle 1
      \draw (0,0) -- (30:1) -- (-30:1) -- (0,0) ;

      \draw [fill] (0,0) circle [radius=.06cm] ;
      \draw [fill] (30:1) circle [radius=.06cm] ;
      \draw [fill] (-30:1) circle [radius=.06cm] ;
      
      %bridge
      \draw (30:1) -- (1.3,1.2) -- (1.9,0.8) ;
      \draw [fill] (1.3,1.2) circle [radius=.06cm] ;
      \draw [fill] (1.9,0.8) circle [radius=.06cm] ;
      
      %triangle 2
      \draw [shift={(1.9,0.8)},rotate=320] (0,0) -- (30:1) -- (-30:1) -- (0,0) ;

      \draw [fill,shift={(1.9,0.8)},rotate=320] (0,0) circle [radius=.06cm] ;
      \draw [fill,shift={(1.9,0.8)},rotate=320] (30:1) circle [radius=.06cm] ;
      \draw [fill,shift={(1.9,0.8)},rotate=320] (-30:1) circle [radius=.06cm] ;
      
      %leaf
      \draw [shift={(1.9,0.8)},rotate=320] (-30:1) -- (-80:0.8) ;
      \draw [fill,shift={(1.9,0.8)},rotate=320] (-80:0.8) circle [radius=0.06cm] ;
      
      %bridge 2
      \draw [fill,shift={(1.9,0.8)},rotate=320] (30:1) -- (1.5,0) ;
      
      %triangle 3
      \draw [rotate=340,shift={(2.915,0.89)}] (0,0) -- (30:1) -- (-30:1) -- (0,0) ;

      \draw [fill,rotate=340,shift={(2.915,0.89)}] (0,0) circle [radius=.06cm] ;
      \draw [fill,rotate=340,shift={(2.915,0.89)}] (30:1) circle [radius=.06cm] ;
      \draw [fill,rotate=340,shift={(2.915,0.89)}] (-30:1) circle [radius=.06cm] ;
      
      %triangle 4
      \draw [shift={(4.03,0.011)},rotate=90] (0,0) -- (30:1) -- (-30:1) -- (0,0) ;

      \draw [fill,shift={(4.03,0.011)},rotate=90] (0,0) circle [radius=.06cm] ;
      \draw [fill,shift={(4.03,0.011)},rotate=90] (30:1) circle [radius=.06cm] ;
      \draw [fill,shift={(4.03,0.011)},rotate=90] (-30:1) circle [radius=.06cm] ;
      
      \node at (2.3,-1.6) {$\Jac(G)\simeq (\Z/3\Z)^4$} ;
    \end{tikzpicture}
    \end{center}

\end{frame}

\begin{frame}
  \frametitle{Future questions}
  \begin{itemize}
    \item Can we get a more precise bound on $k$ such that $(\Z/2\Z)^k\times H$ is not the Jacobian of a simple graph?
    \medskip
      \pause
      \begin{itemize}
      \item Evidence suggests that perhaps $k\ge|H|$. 
      \end{itemize}
      \medskip
      \pause
    \item Can we give a complete description of \emph{exactly} which
      groups are Jacobians of some simple graph?
      \medskip
      \pause
    \item Can we describe which groups are Jacobians of biconnected (wedge-free)
      graphs?
      \medskip
      \pause
      \begin{itemize}
      \item \textbf{Conjecture:} $(\Z/n\Z)^k$ is not the Jacobian of a biconnected graph for all sufficiently large $k$. (maybe for $k\ge n$?)
      \end{itemize}
      \pause
      \item Which \textbf{groups with pairing} occur as Jacobians of graphs?
  \end{itemize}
\end{frame}

\begin{frame}
  \frametitle{Acknowledgements}

  Thanks to our mentors, as well as Sam Payne and the rest of
  SUMRY. 
  
  \vspace{1cm}
  
  Thank you for listening!
  
\end{frame}

%\begin{frame}
  %\frametitle{References}
%\end{frame}

\end{document}
