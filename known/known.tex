\documentclass{amsart}

\title{Facts and Conjectures:\\
  a summer-y SUMRY summary}

\usepackage{amsmath}
\usepackage{amsfonts}
\usepackage{amsthm}
\usepackage{enumitem}
\usepackage{dsfont}
\usepackage{amssymb}
\usepackage{pifont}
\usepackage{mathtools}

\newtheorem{thm}{Theorem}[section]
\newtheorem{lem}[thm]{Lemma}
\newtheorem{prop}[thm]{Proposition}
\newtheorem{conj}[thm]{Conjecture}
\newtheorem{cor}[thm]{Corollary}

\newtheorem*{prop*}{Proposition}

\theoremstyle{definition}
\newtheorem{claim}[thm]{Claim}
\newtheorem{defn}[thm]{Definition}
\newtheorem{quest}[thm]{Question}
\newtheorem{remark}[thm]{Remark}
\newtheorem{fact}[thm]{Fact}
\newtheorem{note}[thm]{Note}

\newtheorem*{claim*}{Claim}
\newtheorem*{quest*}{Question}
\newtheorem*{remark*}{Remark}
\newtheorem*{fact*}{Fact}

\newcommand{\N}{\ensuremath{\mathbb{N}}}
\newcommand{\Z}{\ensuremath{\mathbb{Z}}}
\newcommand{\Q}{\ensuremath{\mathbb{Q}}}
\newcommand{\R}{\ensuremath{\mathbb{R}}}
\newcommand{\C}{\ensuremath{\mathbb{C}}}
\newcommand{\F}{\ensuremath{\mathbb{F}}}
\newcommand{\AP}{\ensuremath{\mathcal{A}_{2^{n}}}}
\newcommand{\BP}{\ensuremath{\mathcal{B}_{2^{n}}}}
\newcommand{\CP}{\ensuremath{\mathcal{C}_{2^{n}}}}
\newcommand{\DP}{\ensuremath{\mathcal{D}_{2^{n}}}}
\newcommand{\EP}{\ensuremath{\mathcal{E}_{2^{n}}}}
\newcommand{\FP}{\ensuremath{\mathcal{F}_{2^{n}}}}

\newcommand{\E}{\ensuremath{\mathbb{E}}}
\newcommand{\1}{\ensuremath{\mathds{1}}}

\newcommand{\pair}[2]{\ensuremath{\langle #1, #2 \rangle}}

\DeclareMathOperator{\Gal}{Gal}
\DeclareMathOperator{\Jac}{Jac}
\DeclareMathOperator{\Var}{Var}
\DeclareMathOperator{\Cov}{Cov}
\DeclareMathOperator{\Div}{Div}
\DeclareMathOperator{\Prin}{Prin}        
\DeclareMathOperator{\im}{im}
\DeclareMathOperator{\val}{val}

\newenvironment{proof_short}
{\emph{Proof idea:}}

\newenvironment{evidence}
{\emph{Evidence:}}

\setcounter{section}{-1}

\begin{document}
\maketitle

\section{Definitions}

\begin{defn}
  $B_n$, the \emph{banana graph on $n$ edges}, or the
  \emph{$n$-banana}, is the graph with $V(B_n) = \{v_1, v_2\}$ and
  edge set consisting of $n$ copies of $\{v_1, v_2\}$.
\end{defn}

\begin{defn}
  Let $C = \{c_1, \ldots, c_m\}$ be a multiset of $m$ positive
  integers. Define the quantities $S(C)$ and $P(C)$ by:

  \begin{align}
    S(C) &= \sum_{i=1}^mc_1 \ldots \hat{c_i} \ldots c_m\\
    P(C) &= \prod_{i=1}^mc_i
  \end{align}

\end{defn}

\begin{defn}
  Let $p$ be a prime, and let $r > 0$. We say $p^r$ is
  \emph{$k$-decomposable} if there exists a finite multiset of
  positive integers $C$ such that $p^r = S(C)$ and $P(C) = k$. Call
  $p^r$ \emph{nonresidue-decomposable} if $p^r$ is $k$-decomposable
  for some $k$ that is a nonresidue modulo $p$.
\end{defn}

\begin{defn}
  $t^*(v,g)$ is the smallest possible number of spanning trees on a
  $2$-edge-connected graph $G$ with $|V(G)| = v$ and genus $g$.
\end{defn}

\section{Facts}

\begin{fact}
  Let $B_n$ be the banana graph on $n$ edges. Then $\Jac(B_n) \simeq
  \Z/n\Z$.
\end{fact}

\begin{proof_short}
  Check that $v_1 - v_2$ generates the Jacobian
\end{proof_short}

\begin{fact}
  Let $B_{p^r}$ be the banana graph on $p^r$ edges, where $p$ is an
  prime. The monodromy pairing on $\Jac(B_{p^r})$ is isomorphic to
  the pairing $\langle \cdot, \cdot \rangle: \Z/p^r\Z \times \Z/p^r\Z
  \to \Q/\Z$ given by

  \begin{equation}
    \langle x, y \rangle  = \frac{xy}{p^r}
  \end{equation}

  for $x,y \in \Z/p^r\Z$.

\end{fact}
\begin{proof_short}
  Use piecewise linear functions to compute the pairing on the
  generator of the group with itself
\end{proof_short}

\begin{fact}
  Let $C = \{c_1, \ldots, c_m\}$ be a finite multiset of positive
  integers, and let $B_C$ be the graph obtained by subdividing the
  $i$th edge of an $m$-banana with $c_i - 1$ vertices (i.e. into $c_i$
  edges). Then $|\Jac(B_C)| = S(C)$.
\end{fact}

\begin{proof_short}
  Make a combinatorial argument to count the number of spanning trees.
\end{proof_short}

\begin{fact}
  Let $C$ be a finite multiset of positive integers. If $S(C) = p^r$
  for some prime $p$, then the monodromy pairing on $\Jac(B_C)$ is
  isomorphic to the pairing on $\Z/p^r\Z$ given by

  \begin{equation}
    \langle x, y \rangle = \frac{P(C)xy}{p^r}
  \end{equation}

  Furthermore, if $P(C)$ is a nonresidue modulo $p$, then this pairing
  is distinct from the pairing $\frac{xy}{p^r}$ given above.
\end{fact}

\begin{proof_short}
  Use a piecewise linear function to compute the pairing of a
  generator with itself.
\end{proof_short}


\begin{fact}
  Let $p$ be an odd prime.
  \begin{enumerate}
  \item Suppose $p$ is not congruent to $1 \pmod 8$. Then $p$ is
    nonresidue-decomposable.

  \item Suppose $p$ is congruent to $1 \pmod 4$. Let $q$ be a prime with
    $p \equiv k \pmod q$ and $\frac{q^2}{4} < p$. If $q$ is
    $k$-decomposable, and $\left( \frac{q}{p} \right) = -1$, then $p$ is
    nonresidue-decomposable.

    In particular, if $\left( \frac{q}{p} \right) = -1$ for some prime $q
    \equiv 3 \pmod 4$ and $\frac{q^2}{4} < p$, then $p$ is
    nonresidue-decomposable.

  \end{enumerate}
\end{fact}

\begin{proof_short}
  Use quadratic reciprocity to construct explicit decompositions in
  the various cases.
\end{proof_short}

\begin{fact}
 Let $p$ be an odd prime, and let $r$ be an integer greater than $1$.
 Then $p^r$ is nonresidue-decomposable.
\end{fact}

\begin{proof_short}
  The decomposition is exactly the same as in the $r=1$ case (possibly
  with a slightly different choice of $q$ if $r$ is even). The
  required bound is somewhat weaker, so in this case we don't need
  GRH.
\end{proof_short}

\begin{fact}
  \label{fact_sum}
  Let $G_1$, $G_2$ be graphs, and let $G_{12}$ be the graph made by
  adjoining $G_1$ and $G_2$ together at a single vertex ($G_1 \wedge G_2
  = G_{12}$). Then

  $$\Jac(G_{12}) \simeq \Jac(G_1) \oplus \Jac(G_2)$$

  where $\Jac(G)$ here denotes the Jacobian of the graph $G$ with the
  associated monodromy pairing, and $\oplus$ denotes the orthogonal sum
  on groups with pairing.
\end{fact}

\begin{proof_short}
  Use the fact that the Laplacian matrix of a wedge of two graphs is
  in block diagonal form, and show that the Smith Normal Form and a
  pseudoinverse of a block diagonal matrix can be found by taking the
  Smith Normal Form and pseudoinverse of the blocks.
\end{proof_short}

\begin{fact}
  Let $G$ be a connected graph, and suppose that $\Jac(G) \simeq
  (\Z/2\Z)^k$ for some $k > 0$. Then there are exactly $k$ pairs of
  vertices in $G$ that have two edges between them, and no pair of
  vertices is connected by more than two edges.

  Furthermore, the simple graph $G'$ obtained by removing duplicate
  edges from $G$ is a tree.
\end{fact}

\begin{proof_short}
  Use the $2$-torsion of the group to show that there can be no more
  than two paths between any pair of vertices, and then use Fact
  \ref{fact_sum} to show that the Jacobian is a direct product of
  Jacobians of cycles.
\end{proof_short}

\begin{fact}
  Let $G$ be a graph with genus $g(G)$. If $\Jac(G)$ contains a
  subgroup isomorphic to $(\Z/p\Z)^k$, then $g(G) \ge k$.
\end{fact}

\begin{proof_short}
  Add edges to a tree one at a time, and use the fact that the
  $p$-rank of the Jacobian changes by at most $1$ when an edge is
  added.
\end{proof_short}

\begin{fact}
  \label{fact_subdivide}
  Let $G$ be a connected graph with genus $g$, and let $G_k$ be
  the graph obtained by subdividing each edge of $G$ by $k-1$ vertices
  (i.e. into $k$ edges). Then $\Jac(G)$ is isomorphic to a subgroup of
  $\Jac(G_k)$.
\end{fact}

\begin{proof_short}
  Show that principal divisors on $G_k$ map to principal divisors on
  $G$ by multiplying the piecewise linear function we have on $G$ by
  $k$.
\end{proof_short}

\begin{fact}
  Let $G$ be a simple $2$-edge-connected graph on $n$ vertices. The
  minimum number of spanning trees on $G$ is $n$, and the minimum is
  achieved when $G$ is a cycle.
\end{fact}

\begin{proof_short}
  If $G$ has genus at least $1$, then we can construct $n$ spanning
  trees by choosing $n$ different edges not to include.
\end{proof_short}

\begin{fact}
  Let $G$ be a graph on $n$ vertices, then $|\Jac(G)|>g$ where $g$ is 
  the genus of $G$.
\end{fact}

\begin{proof_short}
  Induct on $g$ and use the fact that $\tau(G) = \tau(G/e) +
  \tau(G-e)$.  
\end{proof_short}


\begin{fact}
  Let $G$ be a graph with genus $G$, and suppose

  \begin{equation}
    \Jac(G) \simeq
    \Z/a_1\Z \times \Z/a_2\Z \times \ldots \times \Z/a_n\Z
  \end{equation}
  with $a_1 | a_2 | \ldots | a_n$. If $G'$ is the graph obtained from
  $G$ by subdividing each edge of $G$ into $m$ edges, then
  \begin{equation}
    \Jac(G') \simeq \Z/ma_1\Z \times \ldots \Z/ma_n\Z
    \times (\Z/m\Z)^{g - n}
  \end{equation}
\end{fact}
\begin{proof_short}
  Find generators for the invariant factors by putting a $1$ and $-1$
  on adjacent vertices in both the old and new graph.
\end{proof_short}

\begin{fact}
  \label{H_no_2group}
  Let $H$ be a finite abelian group. Then there exists $k_0$
  (depending on $H$) such that for all $k > k_0$, there is no finite
  simple graph $G$ such that

\begin{equation}
  \Jac(G) \simeq (\Z/2\Z)^k \times H
\end{equation}
\end{fact}

\begin{proof_short}
  Relate the max order of an element to the max degree of a vertex,
  and the number of vertices of valency $>2$ to the size of $H$.
\end{proof_short}

\begin{fact}
  Nick's facts about all the pairings on $2$-groups except the
  $E$-pairing
\end{fact}

\begin{proof_short}
  Banana constructions TO THE MAX
\end{proof_short}


\section{Conjectures}


\begin{conj}
  All odd primes are nonresidue-decomposable. (This would imply that
  for any group with pairing $(\Gamma, \langle \cdot , \cdot \rangle)$
  with $\Gamma \simeq \Z/p\Z$, there exists a graph $G$ with $\Jac(G)
  \simeq \Gamma$ and monodromy pairing isomorphic to $\langle \cdot,
  \cdot \rangle$).
\end{conj}
\begin{evidence}
  checked for all primes $ < 10^9$, true for a substantial fraction of
  all primes (anything not 1 modulo 24). True (probably) assuming
  GRH. Almost true unconditionally (bound is $O(p^{1/2 + \epsilon})$,
  need $O(p^{1/2})$.
\end{evidence}

\begin{conj}
  Let $G$ be a graph with genus $g$, and suppose that $\Jac(G) \simeq
  \Z/a_1\Z \times \ldots \times \Z/a_g\Z$ with $a_i | a_{i+1}$ for all
  $0 < i < g$ (i.e., $G$ has $g$ nontrivial invariant factors). Then
  there exists a graph $G'$ such that if each edge of $G'$ is
  subdivided into $a_1$ smaller edges, we obtain $G$.
\end{conj}

\begin{evidence}
  View the Jacobian of $G$ as a lattice on the Jacobian of the
  corresponding metric graph $\Gamma$. Any points of $p$-torsion on
  $\Jac(G)$ should correspond to a subset of the $p^g$ points of
  $p$-torsion on $\Jac(\Gamma)$, and if there are $p^g$ such points in
  $\Jac(G)$ we should be able to ``scale down'' the lattice in some
  way.
\end{evidence}

\begin{conj}
  Nick's conjecture about the $E$-pairing
\end{conj}
\begin{proof_short}
  No idea. Ask Nick.
\end{proof_short}

\section*{Conjectures we don't care about}

\begin{conj}
  Let $G$ be a graph with vertex set $V(G)$. Let $v, w \in V(G)$, and
  let $D \in \Div(G)$ be given by $v - w$. If $[D] \in \Jac(G)$ has
  order $k$, then there at most $k$ independent paths between $v$ and
  $w$ in $G$.
\end{conj}

\begin{evidence}
  Burning algorithm intuition, true for $k = 2$
\end{evidence}

\begin{conj}
  \label{nk_pairs}
  There exist infinitely many pairs of integers $n \ge 3, k \ge 1$
  such that there is no simple graph $G$ where

  \begin{equation}
    \Jac(G) \simeq (\Z/2\Z)^k \times \Z/n\Z
  \end{equation}
\end{conj}

\begin{evidence}
  There doesn't seem to be a simple graph with Jacobian $\Z/2\Z \times
  \Z/4\Z$, and in general it seems hard to construct anything other
  than specific examples of groups with this form (especially if $k >>
  n$).
\end{evidence}

\begin{conj}
  Let $G$ be a simple $2$-edge-connected graph with $n$ spanning
  trees. If $G$ is not a cycle of length $n$, then $|V(G)| \le
  \lfloor\frac{n+1}{2}\rfloor$.
\end{conj}

\begin{evidence}
  Putting a chord in a cycle seems to be the only option here.
\end{evidence}

\begin{conj}
  Let $G$ be a graph, and let $G'$ be the graph that results from
  subdividing each edge of $G$ into two edges.

  Suppose $\Jac(G') \simeq H \times K$ for $H \simeq (\Z/2\Z)^k$, $k
  \ge 2$ and some finite abelian group $K$, and each of the generators
  of $H$ is the equivalence class of some divisor of the form

  $$\sum_{v \in V(G')\setminus V(G)}a_v \cdot v$$ 
  
  Then the monodromy pairing on $\Jac(G')$ restricted to $H$ is
  isomorphic to following pairing on $(\Z/2\Z)^k$:

  $$\langle x, y \rangle = 
  \begin{cases}
    0, & x + y = 0 \\
    \frac{1}{2}, &\textrm{otherwise}
  \end{cases}$$

\end{conj}

\begin{evidence}
  If the vertices that generate the 2-group are on the subdivided
  edges then they must be an even distance apart. Since slope must be
  conserved, and as every new vertex in $G'$ not in $G$ must have
  valency 2, for any function $f$ the difference of $f$ evaluated at
  those vertices must be even. This means that the pairing of an
  element in such a two group with itself is $0$. Since there are only
  two pairings on 2-groups, and only on the hyperbolic pairing is this
  true this must be the hyperbolic pairing.
\end{evidence}

\begin{conj}[Most likely false]

  $t^*(v,g) = \Omega(2^gv)$. 
\end{conj}

\begin{evidence}
Seems to be (sort of) true for small numbers of vertices.

This conjecture is sufficient to prove \ref{nk_pairs} above.
\end{evidence}

\begin{conj}
  If $G$ is a graph with $|V(G)| = v$, genus $g$, and $t^*(v,g)$
  spanning trees, then $G$ is a wedge of graphs isomorphic to either:
  \begin{itemize}
    \item a complete graph
    \item a threshold graph
    \item a cycle graph
  \end{itemize}
\end{conj}

\begin{evidence}
  True for graphs on few vertices ($< 10$)
\end{evidence}


\begin{conj}
  Let $G$ be a 2-vertex-connected simple graph of genus $g$. Then the
  maximum order of any element in $\Jac(G)$ is at most $c \cdot
  \sqrt{g}$ for some constant $c$.  
\end{conj}

\begin{evidence}
  True if the complete graph (or something ``close'' to it) gives the
  ``smallest possible order of elements'' for a given genus.

  If true, this conjecture would give a better bound in the proof of
  fact \ref{H_no_2group} above.
\end{evidence}

\vspace{1cm}
Last updated \today.

\end{document}